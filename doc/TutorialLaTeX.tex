\documentclass[a4paper,12pt,oneside]{book}

%%% Início do preâmbulo %%%
\usepackage[paper=a4paper,top=3cm,bottom=2.5cm,right=2.5cm,left=3cm]{geometry} % https://www.ctan.org/pkg/geometry
\usepackage[brazil]{babel} % https://www.ctan.org/pkg/babel
\usepackage[T1]{fontenc} % https://www.ctan.org/pkg/fontenc
\usepackage[utf8]{inputenc} % https://www.ctan.org/pkg/inputenc (usar no Linux)
%\usepackage[latin1]{inputenc} % (usar no MS Windows ou Mac OS X)
%\usepackage[applemac]{inputenc} % (usar no Macintosh)
\usepackage[12pt]{moresize} % https://ctan.org/pkg/moresize
\usepackage{latexsym} % https://www.ctan.org/pkg/latex-base
\usepackage{textcomp} % https://www.ctan.org/pkg/textcomp
\usepackage{textgreek} % https://www.ctan.org/pkg/textgreek
\usepackage{fontawesome} % https://www.ctan.org/pkg/fontawesome
\usepackage{fixltx2e} % https://www.ctan.org/pkg/fixltx2e
%\usepackage{arev} % https://www.tug.org/FontCatalogue/arev/
%\usepackage{lmodern} % https://www.tug.org/FontCatalogue/latinmodernsans/
%\usepackage{avant} % https://www.ctan.org/pkg/psnfss
\usepackage{colortbl} % https://ctan.org/pkg/colortbl
\usepackage[dvipsnames,table]{xcolor} % https://www.ctan.org/pkg/xcolor
\usepackage[portuges]{datetime2} % https://ctan.org/pkg/datetime2 https://ctan.org/pkg/datetime2-portuges
\usepackage{indentfirst} % https://www.ctan.org/pkg/indentfirst
\usepackage{graphicx} % https://www.ctan.org/pkg/graphicx
\usepackage{wrapfig} % https://www.ctan.org/pkg/wrapfig
\usepackage{eso-pic} % https://www.ctan.org/pkg/eso-pic
\usepackage{sectsty} % https://www.ctan.org/pkg/sectsty
\usepackage{amsmath} % https://www.ctan.org/pkg/amsmath
\usepackage{amssymb} % https://www.ctan.org/pkg/amsfonts
\usepackage{stmaryrd} % https://www.ctan.org/pkg/stmaryrd
\usepackage[stable]{footmisc} % https://www.ctan.org/pkg/footmisc
\usepackage{tikz} % https://www.ctan.org/pkg/pgf
    \usetikzlibrary{matrix,shapes.misc,arrows,chains,trees}
\usepackage{pgfplots} % https://www.ctan.org/pkg/pgfplots
    \pgfplotsset{width=6cm,compat=1.16}
\usepackage{pgfplotstable} % https://www.ctan.org/pkg/pgfplotstable
\usepackage[all]{genealogytree} % https://www.ctan.org/pkg/genealogytree
\usepackage{pifont} % https://www.ctan.org/pkg/pifont
\usepackage{dirtytalk} % https://www.ctan.org/pkg/dirtytalk
\usepackage{paralist} % https://www.ctan.org/pkg/paralist
\usepackage[shortlabels]{enumitem} % https://www.ctan.org/pkg/enumitem
\usepackage{multicol} % https://www.ctan.org/pkg/multicol
\usepackage{multirow} % https://ctan.org/pkg/multirow
\usepackage{ragged2e} % https://www.ctan.org/pkg/ragged2e
\usepackage{vwcol} % https://www.ctan.org/pkg/vwcol
    \setlength{\columnsep}{0.5cm}
\usepackage{setspace} % https://www.ctan.org/pkg/setspace
\usepackage{verbatim} % https://www.ctan.org/pkg/verbatim
\usepackage{array} % https://www.ctan.org/pkg/array
\usepackage{lipsum} % https://www.ctan.org/pkg/lipsum
%\usepackage{blindtext} % https://ctan.org/pkg/blindtext
\usepackage[htt]{hyphenat} % https://ctan.org/pkg/hyphenat
\usepackage{tabularx} % https://www.ctan.org/pkg/tabularx
\usepackage{arydshln} % https://www.ctan.org/pkg/arydshln
\usepackage{imakeidx} % https://www.ctan.org/pkg/imakeidx
\indexsetup{othercode=\ttfamily\small} % comando executado em cada item do índice.
\makeindex[columns=3,title=Índice,intoc,options={-s index_style.ist}] % constrói o índice.
\usepackage{nameref} % https://ctan.org/pkg/nameref
\usepackage[hidelinks,unicode]{hyperref} % https://www.ctan.org/pkg/hyperref
    \hypersetup{
    	colorlinks=false,
    	linkcolor=black,
		urlcolor=black,
		anchorcolor=black,
		citecolor=black,
		filecolor=black,
		linktoc=section,
		pdfstartview=,
		plainpages=false,
		pdfpagelabels=true,
		pdftitle={Tutorial básico de \LaTeX},
		pdfsubject={latex},
		pdfkeywords={tutorial;latex},
		pdfauthor={Daniel Madeira},
		pdfcreator={LaTeX no TeXstudio}, % software que editou o código LaTeX.
		pdfproducer={TeX Live com pdfTeX}, % software que converteu para PDF.
		pdfdisplaydoctitle=true
	}

\usepackage{fancyhdr} % https://www.ctan.org/pkg/fancyhdr
    \fancyhf{}
    \renewcommand{\headrulewidth}{0pt}
    \cfoot{\thepage}
    \pagestyle{fancy}

%\renewcommand{\familydefault}{\sfdefault} % estabelece o uso da fonte sem serifa como padrão.
\renewcommand{\familydefault}{\rmdefault} % estabelece o uso da fonte roman como padrão.

\newcommand*{\justificatt}{%
	\fontdimen2\font=0.4em% interword space
	\fontdimen3\font=0.2em% interword stretch
	\fontdimen4\font=0.1em% interword shrink
	\fontdimen7\font=0.1em% extra space
	\hyphenchar\font=`\-% allowing hyphenation
}

\newcommand{\nomeautor}{Daniel Madeira}
\newcommand{\mesano}{\DTMportugesmonthname{\the\month} de \the\year}

% definição de um novo ambiente, que será usado para montar a folha de rosto.
\newenvironment{folharosto}[1]
	{\begin{center}
		\vspace*{\fill}
		{\LARGE #1}\par
		\vspace{2cm}
    }
    {
		\vspace*{\fill}
		\\\large\nomeautor\\\small\mesano
		\thispagestyle{empty}
		\renewcommand{\thepage}{rosto}
    \end{center}
    }

\newenvironment{moldura}
    {\begin{center}
        \begin{tabular}{|c|}
        \hline\\\hfil
    }
    {
        \\\\\hline
        \end{tabular}
    \end{center}
    }

\title{Tutorial básico de \LaTeX} % define o título do documento.
\author{\nomeautor} % define o autor do documento.
\date{\mesano} % define a data do documento.
%%% Fim do preâmbulo %%%

\begin{document}

\begin{titlepage}
    \AddToShipoutPictureBG*{\includegraphics[width=\paperwidth,height=\paperheight]{clouds.jpg}}
    \maketitle % insere o título, autor e data.
    \thispagestyle{fancy}
    \renewcommand{\thepage}{capa} % troca o número da página pela string capa.
\end{titlepage}

\pagecolor{gray!5!yellow!5} % cor de fundo desta página em diante.

\begin{folharosto}{Tutorial básico de \LaTeX}
	Um tutorial prático com o .tex.
\end{folharosto}

\frontmatter

\tableofcontents % para montar a página do sumário.

\mainmatter

%\counterwithout{section}{chapter} % para quando precisar contar as seções sem incluir o número do capítulo.

\setlength{\parskip}{1em} % define o espaçamento entre os parágrafos.

\chapter*{Prefácio}
\addcontentsline{toc}{chapter}{Prefácio}

O \LaTeX\ é um sistema de preparação de documentos com alta qualidade para composição tipográfica. É utilizado para criar documentos dos mais variados tipos de publicação, como artigos, teses, dissertações, livros, cartas, relatórios ou qualquer outro tipo de documento. Possui um alto grau de exatidão e precisão na diagramação do conteúdo do documento e alta qualidade na formatação automática do documento. O \LaTeX\ é uma ampliação do original sistema de tipografia \TeX. Tornou-se um padrão para produção de documentos científicos.

O sistema \LaTeX\ possui código aberto e é gratuito. Está disponível para qualquer sistema operacional, produzindo o mesmo resultado em qualquer sistema. Cria arquivos pequenos e com resultados de alta qualidade. Ainda, incluem-se ferramentas de exportação do documento para outros formatos, como PostScript e PDF.

Este tutorial tem o propósito de mostrar o mínimo, o básico, para se conseguir produzir um documento, de uma forma prática. O próprio arquivo .tex deste tutorial é um exemplo básico da codificação em \LaTeX. Assim, consulte o código-fonte deste tutorial, em paralelo à sua versão em PDF. Inclusive, alguns recursos de diagramação, que ainda não estão explanados neste tutorial, foram utilizados na montagem deste documento.

\chapter*{Linguagem \LaTeX}
\addcontentsline{toc}{chapter}{Linguagem \LaTeX}

\section*{Arquivo}
\addcontentsline{toc}{section}{Arquivo}

O arquivo do código-fonte em \LaTeX\ deve conter apenas bytes que representem caracteres, sem nenhuma informação adicional. Trata-se do denominado arquivo de texto plano (txt). No sistema \LaTeX, este arquivo recebe a extensão .tex\index{tex@.tex}. A codificação recomendada para este arquivo é a codificação UTF-8 ou Latin1, dependendo do sistema operacional que está processando o \LaTeX.\index{latex@LaTeX}

Pode-se utilizar qualquer editor de texto puro para produzir o arquivo .tex, entretanto, é recomendável utilizar um editor \LaTeX\ para obter uma melhor produtividade. Basicamente, precisa estar instalado no seu computador: uma distribuição \TeX, por exemplo, TeX Live\index{tex live@Tex Live} ou MikTex\index{miktex@MikTex}; e um editor \LaTeX, por exemplo, TeXstudio\index{texstudio@TeXstudio}, Texmaker\index{texmaker@Texmaker} ou TeXworks\index{texworks@TeXworks}.

Mas, atente-se, editar um documento em \LaTeX\ não será WYSIWYG (\textit{What You See Is What You Get})\index{wysiwyg@WYSIWYG}. Será, literalmente, editar o código-fonte do documento, inserindo os comandos de formatação do texto, entre seu conteúdo.

\section*{Estrutura do código}
\addcontentsline{toc}{section}{Estrutura do código}

\subsection*{Partes}
\addcontentsline{toc}{subsection}{Partes}

\begin{multicols}{2}
A estrutura global do código-fonte\\ em \LaTeX\ é, basicamente:
\columnbreak
\begin{verbatim}
	\documentclass{...}
	\begin{document}
	...
	\end{document}
\end{verbatim}
\end{multicols}

A área entre \texttt{\textbackslash documentclass\{...\}} e \texttt{\textbackslash begin\{document\}} é denominada preâmbulo.\index{preambulo@preâmbulo} Neste preâmbulo, ficam os comando que afetam todo o documento. Como as chamadas do uso de pacotes, as definições de parâmetros de comandos, criação de novos comandos etc. A área entre \texttt{\textbackslash begin\{document\}} e \texttt{\textbackslash end\{document\}}\index{document}, após o preâmbulo, forma o bloco principal, o ambiente do documento, onde fica todo o conteúdo do documento.

\subsection*{Ambientes}
\addcontentsline{toc}{subsection}{Ambientes}

Na linguagem \LaTeX, um bloco é definido entre os caracteres \texttt{\{} e \texttt{\}} ou entre os comandos \texttt{\textbackslash begin\{\}}\index{begin@\textbackslash begin} e \texttt{\textbackslash end\{\}}\index{end@\textbackslash end}. Estes comandos formam um bloco de ambiente. Os demais comandos inseridos dentro de um bloco, ou de um ambiente, tem seu efeito restrito ao interior deste bloco, ou ambiente.

No ambiente geral do documento, o processamento dos comandos do sistema \LaTeX\ está em modo texto. Mas, como será visto no capítulo sobre matemática, existe o modo matemático, dentro de um ambiente matemático, onde o sistema \LaTeX\ perfaz um processamento específico.

\subsection*{Caracteres especiais}
\addcontentsline{toc}{subsection}{Caracteres especiais}

Quase tudo pode ser digitado livremente no documento, que fará parte da impressão final, salvo alguns caracteres, que são considerados especiais. Estes caracteres simbólicos são reservados pela linguagem \LaTeX\ porque são para introduzir comandos e possuem um significado especial: \texttt{\# \$ \% \textasciicircum\ \& \_ \{ \} \textasciitilde\ \textbackslash}. Para usar (imprimir) algum destes caracteres no seu texto, digite com o caractere \texttt{\textbackslash} ou use o comando de impressão.

Saiba a função de cada um deles:

\index{\textbackslash \#}
\index{\textbackslash \$}
\index{\textbackslash \%}
\index{textasciicircum@\textbackslash \textasciicircum}
\index{\textbackslash \&}
\index{\textbackslash \_}
\index{\textbackslash \{}
\index{\textbackslash \}}
\index{\textbackslash \~{}}
\index{textasciitilde@\textbackslash textasciitilde}
\index{textbackslash@\textbackslash textbackslash}

\begin{small}
\begin{tabular}{lll}
	Caractere & Função & Como imprimir no PS/PDF\\
	\hline
	\#    & parâmetro de macro  & \texttt{\textbackslash\#}\\
	\$    & modo matemático & \texttt{\textbackslash \$}\\
	\%    & linha de comentário & \texttt{\textbackslash \%}\\
	\^{}  & sobrescrito (no modo matemático) & \texttt{\textbackslash\^{}\{\}} ou \texttt{\textbackslash textasciicircum}\\
	\&    & separador de colunas & \texttt{\textbackslash \&}\\
	\_    & subscrito (no modo matemático) & \texttt{\textbackslash\_}\\
	\{ \} & bloco de processamento & \texttt{\textbackslash\{ \textbackslash\}}\\
	\~{}  & espaço inquebrável & \texttt{\textbackslash textasciitilde} ou \texttt{\textbackslash\~{}\{\}}\\
	\textbackslash & início de comando & \texttt{\textbackslash textbackslash} ou \texttt{\textbackslash}\\
\end{tabular}
\end{small}

\subsection*{Comentário}
\addcontentsline{toc}{subsection}{Comentário}

\begin{multicols}{2}
É possível inserir comentários no código-fonte do arquivo \LaTeX. São informações que não serão processadas ou impressas. Os comentários de uma linha ficam após o caractere \texttt{\%}\index{\%}. Os comentários com mais de uma linha ficam em um bloco de ambiente \texttt{\{comment\}}\index{comment}:
\columnbreak
\begin{small}
\begin{verbatim}
\% comentário de uma linha.

\begin{comment}
    Bloco de comentário
    com mais de uma linha.
\end{comment}
\end{verbatim}
\end{small}
\end{multicols}

\section*{Pacotes}
\addcontentsline{toc}{section}{Pacotes}

O \LaTeX\ inclui alguns comandos básicos, nativos, porém existem muitos outros comandos úteis que são implementados com o uso de pacotes, ativados pelo código-fonte do seu documento. Você poderá usar um pacote no seu documento em \LaTeX\ desde que tenha o respectivo pacote instalado em seu sistema \LaTeX. Assim, para usar e ativar um pacote, inclua o comando \texttt{\textbackslash usepackage\{nomedopacote\}}\index{usepackage@\textbackslash usepackage}. Além do mais, alguns pacotes podem receber parâmetros, por exemplo:
\begin{small}
\begin{verbatim}
   \usepackage[ddmmyyyy]{datetime}
\end{verbatim}
\end{small}

\section*{Comandos}
\addcontentsline{toc}{section}{Comandos}

O \LaTeX\ é uma linguagem movida por comandos (ou macros) no entorno do texto. Os comandos são discriminados pelo caractere \texttt{\textbackslash} e escritos em uma sintaxe como \texttt{\textbackslash comando}.

Alguns comandos possuem duas versões, que são especificadas com a omissão ou o acréscimo do caractere \texttt{*}\index{*} (asterisco). Você verá\footnote{Sempre acompanhe, em paralelo, o arquivo .tex deste tutorial.}, no decorrer deste tutorial, estas duas versões. Nota: Muitos comandos nativos são originais do \TeX\ e, para simplificar, este tutorial generaliza com o nome \LaTeX.

O primeiro comando no código-fonte em \LaTeX\ é o comando \texttt{\textbackslash documentclass}\index{documentclass@\textbackslash documentclass}. Nele se define a classe do documento (ex. article\index{article}, report\index{report}, book\index{book} letter\index{letter}, beamer\index{beamer} ou memoir\index{memoir}) e os parâmetros para tamanho do papel e da fonte, lados de impressão etc.:
\begin{small}\index{a4paper}
\begin{verbatim}
   \documentclass[a4paper,12pt,oneside]{book}
\end{verbatim}
\end{small}

\subsection*{Definição e redefinição}
\addcontentsline{toc}{subsection}{Definição e redefinição}

A linguagem \LaTeX\ permite criar novos comandos, através do comando \texttt{\textbackslash newcommand}\index{newcommand@\textbackslash newcommand}. O novo comando recebe um nome e uma definição. Isto possibilita, por exemplo, criar um comando que retorne um valor, um texto qualquer:
\begin{small}
\begin{verbatim}
   \newcommand{\nome}{valor}
   \newcommand{\agua}{H$_2$O}
\end{verbatim}
\end{small}

Os comandos existentes são redefinidos com o comando \texttt{\textbackslash renewcommand}\index{renewcommand@\textbackslash renewcommand}, exemplo:
\begin{small}\index{contentsname@\textbackslash contentsname}\index{thepage@\textbackslash thepage}
\begin{verbatim}
   \renewcommand*\contentsname{Sumário} % refaz o termo para TOC.
   \renewcommand*{\thepage}{capa} % string capa no número da página
\end{verbatim}
\end{small}

Criar e recriar comandos no \LaTeX\ vai muito além, não limitando-se somente a um valor para a definição. É possível criar combinações de comandos para compôr a definição do novo comando, inclusive a possibilidade de inserção de argumentos. A sintaxe básica é:
\begin{small}
\begin{verbatim}
   \newcommand{\nome}[n]{definição}
\end{verbatim}
\end{small}

O parâmetro \texttt{n} indica o número de argumentos para uso pelo novo comando. Estes argumentos, que em número no máximo podem ser 9, serão invocados por \texttt{\#1}, \texttt{\#2}, \texttt{\#3} etc.

Por exemplo, este novo comando que define um novo modo de inserir os capítulos:
\begin{small}
\begin{verbatim}
   \newcommand{\meucapitulo}[2]{
      \setcounter{chapter}{#1}
      \setcounter{section}{0}
      \chapter*{#2}
      \addcontentsline{toc}{chapter}{#2}
   }
\end{verbatim}
\end{small}

Ou este, por exemplo, que cria um comando para justificar um texto com fonte teletipo (texttt):
\begin{small}
\begin{verbatim}
   \newcommand*{\justificatt}{%
      \fontdimen2\font=0.4em%
      \fontdimen3\font=0.2em%
      \fontdimen4\font=0.1em%
      \fontdimen7\font=0.1em%
      \hyphenchar\font=`\-%
   }
\end{verbatim}
\end{small}

Duas observações:

1º) Perceba o caractere de asterisco em \texttt{\textbackslash newcommand*\{\}} e \texttt{\textbackslash renewcommand*\{\}}. Na origem da linguagem \TeX\ os comandos não podiam ter um \texttt{\textbackslash par} na definição. A linguagem \LaTeX\ controla a possibilidade disso com a ausência ou presença do asterisco na chamada do comando. Com o asterisco, o comando \texttt{\textbackslash newcommand} não aceita parágrafos dentro da definição do comando. São as duas versões destes comandos.

2º) Perceba o caractere \texttt{\%} no fim das linhas. Alguns comandos que lidam precisamente com espaços entre os caracteres, se comportam melhor quando é inserido o \texttt{\%} no final da linha. Provavelmente, uma proteção nos caracteres CR (\textit{carriage return}) e LF (\textit{line feed}).

\subsection*{Definição e redefinição de ambiente}
\addcontentsline{toc}{subsection}{Definição e redefinição de ambiente}

Ambientes também podem ser criados ou redefinidos, com os comandos \texttt{\textbackslash newenvironment}\index{newenvironment@\textbackslash newenvironment} e \texttt{\textbackslash renewenvironment}\index{renewenvironment@\textbackslash renewenvironment}, da mesma forma dos novos comandos. A sintaxe é semelhante, veja um exemplo:
\begin{multicols}{2}
\begin{small}
\begin{verbatim}
\newenvironment{moldura}
   {\begin{center}
      \begin{tabular}{|c|}
      \hline\\\hfil
   }
   {
      \\\\\hline
      \end{tabular}
      \end{center}
   }
\end{verbatim}
\end{small}
\columnbreak
Usando desta forma:
\begin{small}
\begin{verbatim}
\begin{moldura}
   texto emoldurado
\end{moldura}
\end{verbatim}
\end{small}
Imprime isto:
\begin{moldura}
texto emoldurado
\end{moldura}
\end{multicols}

\subsection*{Uso}
\addcontentsline{toc}{subsection}{Uso}

A sintaxe geral de uso de um comando é:
\begin{small}
\begin{verbatim}
   \comando[argumento opcional]{argumento compulsório}
\end{verbatim}
\end{small}

O nome do comando é sensível a letras maiúsculas e minúsculas e compõe somente de caracteres alfa-numéricos. Os argumentos podem ser mais de um, se houver.

\subsection*{Unidades de medida}
\addcontentsline{toc}{subsection}{Unidades de medida}

As unidades de medidas aceitas nos comandos do LaTeX são:

\begin{tabular}{>{\ttfamily}ll}
pt\index{pt} & pontos (1/72 polegadas)\\
mm\index{mm} & milímetros\\
cm\index{cm} & centímetros\\
in\index{in} & polegadas\\
ex\index{ex} & altura de um x minúsculo na fonte corrente\\
em\index{em} & largura de um M maiúsculo na fonte corrente\\
mu\index{mu} & unidade matemática igual à 1/18em\\
\end{tabular}

Obs: Muitos comandos aceitam valores negativos, por exemplo \texttt{\textbackslash hspace\{-1.5em\}}.

\subsection*{Cores}
\addcontentsline{toc}{subsection}{Cores}

Com o uso do pacote \texttt{\{xcolor\}}\index{xcolor} é possível definir cores para o texto, fundo do texto, fundo da página, linhas e colunas de tabelas, gráficos etc. Pode-se usar as cores pré-definidas ou definir novas cores usando valores em RGB, Hex ou CMYK. Inicialmente, com o uso do pacote \texttt{\{xcolor\}}, existem algumas cores pré-definidas, que são:

\texttt{\small\justificatt\color{blue} black, blue, brown, cyan, darkgray, gray, green, lightgray, lime, magenta, olive, orange, pink, purple, red, teal, violet, white, yellow.}

Se o pacote foi carregado com a opção \texttt{[dvipsnames]}\index{dvipsnames}, então um total de 68 cores estarão pré-definidas:

\texttt{\small\justificatt\color{OliveGreen} Apricot, Aquamarine, Bittersweet, Black, Blue, BlueGreen, BlueViolet, BrickRed, Brown, BurntOrange, CadetBlue, CarnationPink, Cerulean, CornflowerBlue, Cyan, Dandelion, DarkOrchid, Emerald, ForestGreen, Fuchsia, Goldenrod, Gray, Green, GreenYellow, JungleGreen, Lavender,
LimeGreen, Magenta, Mahogany, Maroon, Melon, MidnightBlue, Mulberry, NavyBlue, OliveGreen, Orange, OrangeRed, Orchid, Peach, Periwinkle, PineGreen, Plum, ProcessBlue, Purple, RawSienna, Red, RedOrange, RedViolet, Rhodamine, RoyalBlue, RoyalPurple, RubineRed, Salmon, SeaGreen, Sepia, SkyBlue, SpringGreen, Tan, TealBlue, Thistle, Turquoise, Violet, VioletRed, White, WildStrawberry, Yellow, YellowGreen, YellowOrange.}

Para definir novas cores, insira os comandos no preâmbulo, seguindo estes exemplos:\index{definecolor@\textbackslash definecolor}
\begin{footnotesize}
\begin{verbatim}
   \definecolor{cinza}{gray}{0.95}
   \definecolor{laranja}{RGB}{255,127,0}
   \definecolor{laranja}{HTML}{FF7F00}
   \definecolor{laranja}{cmyk}{0,0.5,1,0}
\end{verbatim}
\end{footnotesize}

Ou ainda, crie uma mistura de cores, por exemplo:
\index{colorlet@\textbackslash colorlet}
\begin{small}
\begin{verbatim}
   \colorlet{azurelo}{blue!50!yellow}
\end{verbatim}
\end{small}

Para cada lugar de uso das cores, um comando específico será utilizado, mas a referência à cor será a mesma. Por exemplo, o texto é colorizado com \texttt{\textbackslash textcolor\{\textit{cor}\}}, uma linha de tabela com \texttt{\textbackslash rowcolor\{\textit{cor}\}} e por aí vai. Em todos estes comandos, a cor é referenciada pelo seu nome no argumento do comando.

A cor pode ser implementada integralmente, com 100\% de sua intensidade, ou reduzida em sua intensidade ou até misturada com outras cores. Para a redução de intensidade usa-se a sintaxe com seu \texttt{nome + exclamação + valor}, por exemplo: \texttt{blue!60}. A mistura de cores funciona acrescentando mais uma exclamação e o nome da segunda cor, que também pode ter sua intensidade reduzida, por exemplo: \texttt{blue!60!yellow}. 

Um exemplo de comando que usa uma cor:

\texttt{\small\textbackslash textcolor\{red!50!violet!90\}\{\textcolor{red!50!violet!90}{seu texto}\}}

\newgeometry{margin=4.5cm,nohead} % nova disposição desta página em diante.

\chapter*{Formatação da página}
\addcontentsline{toc}{chapter}{Formatação da página}

\section*{Margens}
\addcontentsline{toc}{section}{Margens}

O pacote \texttt{\{geometry\}}\index{geometry} proporciona um meio de configurar a disposição da página. Por exemplo, esta página está com a margem de 4,5cm e sem o espaço do cabeçalho (veja o código-fonte deste tutorial).
\begin{small}
\begin{verbatim}
   \newgeometry{margin=4.5cm,nohead}
\end{verbatim}
\end{small}

Basicamente usa-se dois comandos, um para definir e outro para restaurar o que foi definido no preâmbulo:
\index{newgeometry@\textbackslash newgeometry}
\index{restoregeometry@\textbackslash restoregeometry}
\begin{small}
\begin{verbatim}
   \newgeometry{top=1.5cm,bottom=1.5cm,right=1cm,left=1cm}

   \restoregeometry
\end{verbatim}
\end{small}

Ambos os comandos implicam em uma quebra de página, para fazer valer a alteração na dimensão.

\restoregeometry % restaura a geometria definida no carregamento do pacote. Irá ter uma quebra de página aqui.

\section*{Quebras de página}
\addcontentsline{toc}{section}{Quebras de página}

Além dos comandos de geometria de página, há outros específicos para impor uma quebra na página. Basicamente são os comandos:
\begin{small}
\begin{verbatim}
   \pagebreak
   \newpage
   \clearpage
\end{verbatim}
\end{small}

O comando \texttt{\textbackslash pagebreak}\index{pagebreak@\textbackslash pagebreak} faz com que os parágrafos se desloquem para preencher toda página, para não deixar espaço vazio no fim. Diferentemente, o comando \texttt{\textbackslash newpage}\index{newpage@\textbackslash newpage} não estica os espaços entre os parágrafos, deixando um grande espaço vazio no fim da página. O comando \texttt{\textbackslash clearpage}\index{clearpage@\textbackslash clearpage} é similar ao \texttt{\textbackslash newpage}, apenas agindo também nas figuras.

\subsection*{Mesma página}
\addcontentsline{toc}{subsection}{Mesma página}

Caso tenha um conteúdo que queira manter-se em uma mesma página, sem quebra pelo meio, use o ambiente \texttt{\{samepage\}}\index{samepage}.
\begin{small}
\begin{verbatim}
   \begin{samepage}
   ...
   \end{samepage}
\end{verbatim}
\end{small}

\section*{Espaço vazio}
\addcontentsline{toc}{section}{Espaço vazio}

É possível adicionar espaços vazios entre os conteúdos na página. Basicamente, existe os comandos para espaço horizontal, que empurra o próximo conteúdo horizontalmente, e para espaço vertical, que empurra o próximo conteúdo verticalmente (valores negativos realizam o inverso, contraem o espaço). Os comandos são:
\index{hspace@\textbackslash hspace}
\index{vspace@\textbackslash vspace}
\begin{small}
\begin{verbatim}
   \hspace{medida}
   \vspace{medida}
\end{verbatim}
\end{small}

Exemplos:
\begin{small}
\begin{verbatim}
   \hspace{1.5em}
   \vspace{4cm}
\end{verbatim}
\end{small}

Quando for lidar com letras e palavras, use a unidade de medida \texttt{em}, pois esta unidade é proporcional ao tamanho e família da fonte do texto.

Uma interessante utilidade, caso queira posicionar os primeiros parágrafos no topo da página e os últimos parágrafos no fim da página, use um destes dois comandos equivalentes para esticar o espaço vazio entre eles:\index{vfill@\textbackslash vfill}
\begin{small}
\begin{verbatim}
   \vspace{\fill}

   \vfill
\end{verbatim}
\end{small}

\vspace{3cm} % insere um espaço vertical vazio.

Por exemplo, este parágrafo foi para baixo com \texttt{\textbackslash vspace\{3cm\}}. Obs.: Usando \texttt{\textbackslash vspace*} (com asterisco) o \LaTeX\ não remove o espaço vertical do fim da página (a documentação oficial também é confusa nesta explicação).

\section*{Estilo}
\addcontentsline{toc}{section}{Estilo}

Para limpar o estilo aplicado, na página atual ou nas próximas páginas, use um destes comandos com o estilo desejado:
\index{thispagestyle@\textbackslash thispagestyle}
\index{pagestyle@\textbackslash pagestyle}
\begin{small}
\begin{verbatim}
   \thispagestyle{empty}

   \pagestyle{plain}
\end{verbatim}
\end{small}

O estilo empty\index{empty} limpa tanto o cabeçalho quanto o rodapé. No estilo plain\index{@plain}, que é o padrão, o cabeçalho fica vazio e o rodapé contém o número da página no centro.

\section*{Colorindo}
\addcontentsline{toc}{section}{Colorindo}

Uma ou mais páginas podem ser coloridas com o comando \texttt{\textbackslash pagecolor\{\}}\index{pagecolor@\textbackslash pagecolor}. Este comando terá efeito da página atual em diante. Por exemplo:
\begin{small}
\begin{verbatim}
   \pagecolor{gray!10!yellow!10}
\end{verbatim}
\end{small}

Para cancelar a definição da cor na página atual em diante, use o comando:
\index{nopagecolor@\textbackslash nopagecolor}
\begin{small}
\begin{verbatim}
   \nopagecolor
\end{verbatim}
\end{small}

\chapter*{Seções no documento}
\addcontentsline{toc}{chapter}{Seções no documento}

\section*{Níveis de seção}
\addcontentsline{toc}{section}{Níveis de seção}

O documento em \LaTeX\ pode ser seccionado em até 7 níveis, dependendo da classe declarada. As divisões de conteúdo no documento podem ser:

\index{part@\textbackslash part}
\index{chapter@\textbackslash chapter}
\index{section@\textbackslash section}
\index{subsection@\textbackslash subsection}
\index{subsubsection@\textbackslash subsubsection}
\index{paragraph@\textbackslash paragraph}
\index{subparagraph@\textbackslash subparagraph}

\begin{tabular}{lrcl}
	\textbf{Divisão} & \textbf{Nível} & & \textbf{Comando}\\
	\hline
	parte         & -1 & & \texttt{\textbackslash part\{\textit{nome}\}}\\
	capítulo      & 0  & & \texttt{\textbackslash chapter\{\textit{nome}\}}\\
	seção         & 1  & & \texttt{\textbackslash section\{\textit{nome}\}}\\
	subseção      & 2  & & \texttt{\textbackslash subsection\{\textit{nome}\}}\\
	subsubseção   & 3  & & \texttt{\textbackslash subsubsection\{\textit{nome}\}}\\
	parágrafo     & 4  & & \texttt{\textbackslash paragraph\{\textit{nome}\}}\\
	subparágrafo  & 5  & & \texttt{\textbackslash subparagraph\{\textit{nome}\}}\\
	\hline
\end{tabular}

Basta usar o comando do nível desejado e o que vier depois será desta divisão. Não há comando de encerramento, o próximo comando da próxima divisão é que indica a mudança. Exemplo:
\begin{small}
\begin{verbatim}
    \begin{document}
    
        \chapter{Introdução}
        ...
        \chapter{Materiais}
            \section{Líquidos}
            ...
            \section{Sólidos}
                \subsection{Descartáveis}
                ...
                \subsection{Não-descartáveis}
                ...
        \chapter{Conclusão}
        ...

    \end{document}
\end{verbatim}
\end{small}

\newpage
Os comandos destes níveis podem ser escritos na sintaxe sem o caractere \texttt{*}, desta forma, são numerados, prefixados com o número e adicionados automaticamente no sumário do documento. Com a utilização do \texttt{*}, logo após o nome do comando, nada disso acontece.

\section*{Livro}
\addcontentsline{toc}{section}{Livro}

Em documentos da classe \texttt{book}\index{book}, opcionalmente pode-se seccionar o conteúdo em quatro partes: frontal, principal, apêndice e traseira.

Tradicionalmente, a parte frontal contém a página do título, folha de rosto, resumo, sumário, prefácio, lista de figuras e lista de tabelas. A parte principal contém o conteúdo propriamente dito. Logo após existe o apêndice e a parte traseira contém o glossário, notas, bibliografia e índice.

São quatro comandos que definem estas partes:
\begin{small}
\begin{verbatim}
   \frontmatter
   \mainmatter
   \appendix
   \backmatter
\end{verbatim}
\end{small}

A parte em \texttt{\textbackslash frontmatter}\index{frontmatter@\textbackslash frontmatter} terá a numeração romana nas páginas e não terá os capítulos numerados. A parte em \texttt{\textbackslash mainmatter}\index{mainmatter@\textbackslash mainmatter} terá o comportamento padrão do documento e a sequencia numérica das páginas é reiniciada. A parte \texttt{\textbackslash appendix}\index{appendix@\textbackslash appendix} reinicia a numeração de capítulos, usa letras na numeração de capítulos e continua seguindo a numeração das páginas principais. A parte \texttt{\textbackslash backmatter}\index{backmatter@\textbackslash backmatter} continua seguindo a numeração das páginas principais mas volta a desativar a numeração dos capítulos.

\chapter*{Formatação de texto}
\addcontentsline{toc}{chapter}{Formatação de texto}

\section*{Parágrafo}
\addcontentsline{toc}{section}{Parágrafo}

\subsection*{Quebras de parágrafo e linha}
\addcontentsline{toc}{subsection}{Quebras de parágrafo e linha}

Isto é um texto em um parágrafo. O alinhamento justificado é aplicado por padrão. A linha se estica horizontalmente para ocupar todo o espaço entre as margens. Se for preciso, o \LaTeX\ aplica a hifenização das palavras. A regra da língua portuguesa vem com o uso do pacote \texttt{\{babel\}}.

Para iniciar um novo parágrafo basta pular uma linha no código-fonte do \LaTeX.

Ou usar o comando \texttt{\textbackslash par}\index{par@\textbackslash par} no final da linha; \par Para quebrá-la em um novo parágrafo.

Isto é um texto em uma linha, \newline o comando \texttt{\textbackslash newline}\index{newline@\textbackslash newline} ou \texttt{\textbackslash\textbackslash}\index{\textbackslash \textbackslash} (duas barras invertidas) faz uma quebra de linha, sem criar um novo parágrafo.

\subsection*{Espaçamento entre parágrafos}
\addcontentsline{toc}{subsection}{Espaçamento entre parágrafos}

Os parágrafos, por padrão, não possuem um espaçamento entre eles distinto da separação simples entre as linhas. Use esta combinação de comandos para definir um espaçamento dos próximos parágrafos:

\index{setlength@\textbackslash setlength}
\index{parskip@\textbackslash parskip}
\begin{small}
\begin{verbatim}
   \setlength{\parskip}{1em}
\end{verbatim}
\end{small}

\setlength{\parskip}{1em} % define um espaçamento entre os parágrafos (padrão é 0).

Perceba que este parágrafo já possui \texttt{1em} de distância do parágrafo anterior e também do próximo parágrafo abaixo.

De agora em diante, todos os parágrafos terão este espaçamento entre eles. Obs.: Os espaços entre os blocos de ambiente costumam ter outro tamanho, geralmente maior.

\subsection*{Espaçamento entre linhas}
\addcontentsline{toc}{subsection}{Espaçamento entre linhas}

Por padrão, ocorre o espaçamento simples entre as linhas. Alguns comandos modificam isso, do pacote \texttt{\{setspace\}}:

\index{onehalfspacing@\textbackslash onehalfspacing}
\index{doublespacing@\textbackslash doublespacing}
\index{singlespacing@\textbackslash singlespacing}
\index{baselinestretch@\textbackslash baselinestretch}
\index{normalsize@\textbackslash normalsize}
\begin{small}
\begin{verbatim}
   \onehalfspacing
   \doublespacing
   \singlespacing

   \renewcommand{\baselinestretch}{0.80}\normalsize
   \renewcommand{\baselinestretch}{1}\normalsize
\end{verbatim}
\end{small}

Seguem os quatro exemplos para espaçamento entre as linhas de 0,80, simples(1), 1,5 e duplo:

\renewcommand{\baselinestretch}{0.80}\normalsize % outra forma de definir o espaçamento.
\lipsum[11]

\singlespacing % espaçamento simples entre as linhas (padrão).
\lipsum[11]

\onehalfspacing % espaçamento de 1,5 entre as linhas
\lipsum[11]

\doublespacing % espaçamento duplo entre as linhas.
\lipsum[11]

\renewcommand{\baselinestretch}{1}\normalsize % valor 1 é o simples.

\newpage

\subsection*{Espaçamento entre palavras}
\addcontentsline{toc}{subsection}{Espaçamento entre palavras}

Além do espaço comum, digitado pela tecla de espaço do teclado, entre as palavras, existem alguns comandos que alteram o espaço para mais ou menos, em uma largura fixa. Estes comandos também servem para forçar a colocação de espaço onde a formatação automática do \LaTeX\ suprime\footnote{Aqui mesmo ocorreu isso, veja no código-fonte.}.

\index{\textbackslash ,}
\index{\textbackslash :}
\index{\textbackslash ;}
\index{\textbackslash "!}
\index{thinspace@\textbackslash thinspace}
\index{medspace@\textbackslash medspace}
\index{thickspace@\textbackslash thickspace}
\index{negthinspace@\textbackslash negthinspace}
\index{negmedspace@\textbackslash negmedspace}
\index{negthickspace@\textbackslash negthickspace}
\index{quad@\textbackslash quad}
\index{qquad@\textbackslash qquad}

\begin{tabular}{lll}
\multicolumn{2}{l}{\textbf{Comando curto e longo}} & \textbf{Tamanho}\\
\hline
\texttt{\textbackslash ,} & \texttt{\textbackslash thinspace} & 3/18 de \texttt{\textbackslash quad} (3 mu)\\
\texttt{\textbackslash :} & \texttt{\textbackslash medspace} & 4/18 de \texttt{\textbackslash quad} (4 mu)\\
\texttt{\textbackslash ;} & \texttt{\textbackslash thickspace} & 5/18 de \texttt{\textbackslash quad} (5 mu)\\
\texttt{\textbackslash !} & \texttt{\textbackslash negthinspace} & -3/18 de \texttt{\textbackslash quad} (-3 mu)\\
 & \texttt{\textbackslash negmedspace} & -4/18 de \texttt{\textbackslash quad} (-4 mu)\\
 & \texttt{\textbackslash negthickspace} & -5/18 de \texttt{\textbackslash quad} (-5 mu)\\
\texttt{\textbackslash } {\footnotesize (espaço após a barra)} &  & espaço normal\\
 & \texttt{\textbackslash quad} & espaço da fonte corrente (18 mu)\\
 & \texttt{\textbackslash qquad} & dobro de \texttt{\textbackslash quad} (36 mu)\\
\end{tabular}

Obs.: O espaço comum é um espaço expansível, isto é, sua largura é flexível para esticar ou contrair, quando o parágrafo possui o alinhamento justificado. Os comandos acima produzem um espaço não expansível, com largura fixa. Como alternativa, o comando \texttt{\textbackslash space}\index{space@\textbackslash space} produz um espaço que é expansível.

\subsection*{Indentação}
\addcontentsline{toc}{subsection}{Indentação}

Os parágrafos costumam indentar-se automaticamente. Mas, em uma instalação padrão do \LaTeX, o primeiro não se indenta, somente do segundo em diante. Neste documento, se não tivesse instalado o pacote \texttt{\{indentfirst\}}\index{indentfirst}, este primeiro parágrafo não estaria indentado.

Para definir um tamanho de indentação, usa-se: \texttt{\textbackslash setlength\{\textbackslash parindent\}\{3em\}}\index{setlength@\textbackslash setlength}\index{parindent@\textbackslash parindent}

\setlength{\parindent}{3em} % define a largura da indentação (1.5em é o padrão).
Por exemplo, este parágrafo está com 3em de tamanho na indentação.

\noindent
Já este parágrafo não está indentado, pois antes dele há o comando \texttt{\textbackslash noindent}\index{noindent@\textbackslash noindent}

\setlength{\parindent}{1.5em}
Voltando à indentação normal, com: \texttt{\textbackslash setlength\{\textbackslash parindent\}\{1.5em\}}

\indent O comando \texttt{\textbackslash indent}\index{indent@\textbackslash indent} força uma indentação, caso não ocorra, mas só se \texttt{\textbackslash parindent} estiver diferente de zero.

Obs.: O comando \texttt{\textbackslash hspace\{1.5em\}} colocado no início de um parágrafo simula o mesmo efeito da indentação, já o valor negativo, \texttt{\textbackslash hspace\{-1.5em\}}, anula a indentação existente.

\newpage
\subsection*{Alinhamento}
\addcontentsline{toc}{subsection}{Alinhamento}

Além do alinhamento justificado, que é o padrão, outros alinhamentos podem ser aplicados com os comandos \texttt{\textbackslash begin ... \textbackslash end}. Como já visto, estes comandos criam um ambiente de formatação de bloco de texto.

\index{center}
\index{flushleft}
\index{flushright}

\begin{center}
	Conteúdo centralizado na página,\\ definido com \texttt{\textbackslash begin\{center\} ...  \textbackslash end\{center\}}. 
\end{center}
\vspace{-1.5em}
\begin{flushleft}
	Conteúdo alinhado à esquerda,\\ definido com \texttt{\textbackslash begin\{flushleft\} ... \textbackslash end\{flushleft\}}.
\end{flushleft}
\vspace{-1.5em}
\begin{flushright}
	Conteúdo alinhado à direita,\\ definido com \texttt{\textbackslash begin\{flushright\} ... \textbackslash end\{flushright\}}.
\end{flushright}

\section*{Fontes}
\addcontentsline{toc}{section}{Fontes}

\subsection*{Tamanho}
\addcontentsline{toc}{subsection}{Tamanho}

Existem alguns tamanhos para os caracteres da fonte (e incrementado pelo pacote \texttt{\{moresize\}})\index{moresize}, a partir do normal definido entre 10pt, 11pt ou 12pt, os quais são ativados por comandos com sintaxe inline ou para bloco de ambiente:

\index{tiny@\textbackslash tiny}
\index{ssmall@\textbackslash ssmall}
\index{scriptsize@\textbackslash scriptsize}
\index{footnotesize@\textbackslash footnotesize}
\index{small@\textbackslash small}
\index{normalsize@\textbackslash normalsize}
\index{large@\textbackslash large}
\index{Large@\textbackslash Large}
\index{LARGE@\textbackslash LARGE}
\index{huge@\textbackslash huge}
\index{Huge@\textbackslash Huge}
\index{HUGE@\textbackslash HUGE}

\begin{tabular}{l>{\ttfamily}r}
	\tiny{Texto miúdo} & \textbackslash tiny\{\}\\
	\ssmall{Texto ppequeno} & \textbackslash ssmall\{\}\\
	\scriptsize{Texto script} & \textbackslash scriptsize\{\}\\
	\footnotesize{Texto rodapé} & \textbackslash footnotesize\{\}\\
	\small{Texto pequeno} & \textbackslash small\{\}\\
	\normalsize{Texto normal} & \textbackslash normalsize\{\}\\
	\large{Texto largo} & \textbackslash large\{\}\\
	\Large{Texto Largo} & \textbackslash Large\{\}\\
	\LARGE{Texto LARGO} & \textbackslash LARGE\{\}\\
	\huge{Texto imenso} & \textbackslash huge\{\}\\
	\Huge{Texto Imenso} & \textbackslash Huge\{\}\\
	\HUGE{Texto IMENSO} & \textbackslash HUGE\{\}\\
\end{tabular}

A forma da sintaxe inline também pode ser assim: \texttt{\{\textbackslash small ... \}}.

\begin{normalsize}
Outra maneira, de definir o tamanho do texto, é em um ambiente com bloco de parágrafo. Exemplo:

\begin{small}
\texttt{\textbackslash begin\{large\}\\
\indent...\\
\indent\textbackslash end\{large\}}.
\end{small}
\end{normalsize}

\subsection*{Estilo}
\addcontentsline{toc}{subsection}{Estilo}

Os diversos estilos para os caracteres são:

\index{textbf@\textbackslash textbf}
\index{textmd@\textbackslash textmd}
\index{textit@\textbackslash textit}
\index{textsl@\textbackslash textsl}
\index{underline@\textbackslash underline}
\index{textsuperscript@\textbackslash textsuperscript}
\index{textsubscript@\textbackslash textsubscript}
\index{text@\textbackslash text}
\index{textnormal@\textbackslash textnormal}
\index{textup@\textbackslash textup}
\index{textsc@\textbackslash textsc}
\index{bfseries@\textbackslash bfseries}
\index{mdseries@\textbackslash mdseries}
\index{itshape@\textbackslash itshape}
\index{normalfont@\textbackslash normalfont}
\index{upshape@\textbackslash upshape}
\index{scshape@\textbackslash scshape}

\begin{tabular}{l>{\ttfamily}r>{\ttfamily}l}
    \textbf{Texto em negrito} & \textbackslash textbf\{\} & \textbackslash bfseries\\
    \textmd{Texto médio} & \textbackslash textmd\{\} & \textbackslash mdseries\\
    \textit{Texto em itálico} & \textbackslash textit\{\} & \textbackslash itshape\\
    \textbf{\textit{Texto em negrito e itálico}} & \textbackslash textbf\{\textbackslash textit\{\}\}\\
    \textsl{Texto inclinado} & \textbackslash textsl\{\} & \textbackslash slshape\\
    \underline{Texto sublinhado} & \textbackslash underline\{\}\\
    \textsuperscript{Texto sobrescrito} & \textbackslash textsuperscript\{\}\\
    \textsubscript{Texto subscrito} & \textbackslash textsubscript\{\}\\
    \text{Texto normal} & \textbackslash text\{\}\\
    \textnormal{Texto normal} & \textbackslash textnormal\{\} & \textbackslash normalfont\\
    \textup{Texto vertical} & \textbackslash textup\{\} & \textbackslash upshape\\
    \textsc{Texto em pequenas maiúsculas} & \textbackslash textsc\{\} & \textbackslash scshape\\
\end{tabular}

\subsection*{Família}
\addcontentsline{toc}{subsection}{Família}

As famílias de caracteres na fonte em uso basicamente são:

\index{textrm@\textbackslash textrm}
\index{texttt@\textbackslash texttt}
\index{textsf@\textbackslash textsf}
\index{rmfamily@\textbackslash rmfamily}
\index{ttfamily@\textbackslash ttfamily}
\index{sffamily@\textbackslash sffamily}

\begin{tabular}{l>{\ttfamily}r>{\ttfamily}l}
	\textrm{Texto romano (serifado).} & \textbackslash textrm\{\} & \textbackslash rmfamily\\
	\texttt{Texto de máquina (monoespaçado).} & \textbackslash texttt\{\} & \textbackslash ttfamily\\
	\textsf{Texto sem serifa.} & \textbackslash textsf\{\} & \textbackslash sffamily\\
\end{tabular}

Porém, nem todas as fontes suportam esta variação de família. Por exemplo, a fonte Arev é exclusivamente do tipo sem serifa.

\subsection*{Colorindo}
\addcontentsline{toc}{subsection}{Colorindo}

Para colorir um texto, usa-se o comando \texttt{\textbackslash textcolor\{\textit{cor}\}\{...\}}\index{textcolor@\textbackslash textcolor} ou \texttt{\{\textbackslash color\{\textit{cor}\} ...\}}\index{color@\textbackslash color}. Ou, pode-se definir uma cor diretamente no comando. Exemplos:

\texttt{\small\textbackslash textcolor\{brown!70!black\}\{\textcolor{brown!70!black}{seu texto}\}}

\texttt{\{\small\textbackslash color\{brown!70!black\} {\color{brown!70!black} seu texto}\}}

\texttt{\small\textbackslash textcolor[RGB]\{190,85,0\}\{\textcolor[RGB]{190,85,0}{seu texto}\}}

\texttt{\{\small\textbackslash color[RGB]\{190,85,0\} {\color[RGB]{190,85,0} seu texto}\}}

Para o fundo do texto, usa-se o comando \texttt{\textbackslash colorbox\{\textit{cor}\}\{...\}}\index{colorbox@\textbackslash colorbox}:

\texttt{\small\textbackslash colorbox\{Sepia!10\}\{\colorbox{Sepia!10}{seu texto}\}}

\section*{Estrutura de texto}
\addcontentsline{toc}{section}{Estrutura de texto}

\subsection*{Tabela}
\addcontentsline{toc}{subsection}{Tabela}

Uma tabela pode ser construída com o ambiente \texttt{\{tabular\}}\index{tabular} ou com o ambiente \texttt{\{tabbing\}}\index{tabbing}. O ambiente \texttt{\{tabular\}} requer um argumento que indica quantas colunas terá e qual o alinhamento de cada coluna. Já no ambiente \texttt{\{tabbing\}}, as colunas assim como suas larguras são definidas diretamente pelos separadores.

No ambiente \texttt{\{tabular\}}, para indicar a quantidade de colunas e o respectivo alinhamento, use a quantidade de letras \texttt{l} (alinhamento à esquerda), \texttt{c} (alinhamento ao centro) e \texttt{r} (alinhamento à direita). No conteúdo, as colunas são delimitadas pelo caractere \texttt{\&}. Já no ambiente \texttt{\{tabbing\}}, os separadores das colunas serão os comandos \texttt{\textbackslash =}\index{\textbackslash =} ou \texttt{\textbackslash >}\index{\textbackslash >} etc. (veja lista). Em ambos os ambientes, ao final de cada linha, use uma quebra.

Estrutura básica de construção de tabela:

\noindent
\begin{minipage}[t]{0.5\linewidth}
\begin{small}
\begin{verbatim}
   \begin{tabular}{lccr}
   11 & 12 & 13 & 14 \\
   21 & 22 & 23 & 24 \\
   31 & 32 & 33 & 34 \\
   41 & 42 & 43 & 44 \\
   51 & 52 & 53 & 54 \\
   \end{tabular}
\end{verbatim}
\end{small}
\end{minipage}\hspace{\fill}
\begin{minipage}[t]{0.5\linewidth}
\begin{small}
\begin{verbatim}
   \begin{tabbing}
   11 \= 12 \= 13 \= 14 \\
   21 \= 22 \= 23 \= 24 \\
   31 \= 32 \= 33 \= 34 \\
   41 \= 42 \= 43 \= 44 \\
   51 \= 52 \= 53 \= 54 \\
   \end{tabbing}
\end{verbatim}
\end{small}
\end{minipage}

Além do \texttt{\{l c r\}} no argumento, ainda há \texttt{p\{largura\}}, \texttt{m\{largura\}} e \texttt{b\{largura\}}, para indicar um parágrafo na coluna com alinhamento vertical no topo, meio e embaixo respectivamente. Estes argumentos definem a largura fixa da coluna.
\index{kill@\textbackslash kill}

\noindent
\begin{minipage}[t]{0.1\linewidth}
Separadores do\\ \texttt{\{tabbing\}}:
\end{minipage}\hspace{1.5cm}
\noindent
\begin{minipage}[t]{0.3\linewidth}
\begin{small}
\begin{verbatim}
\= (tabulação normal)
\> (avança tabulação)
\< (à esquerda da margim)
\+ (move margem pra direita)
\end{verbatim}
\end{small}
\end{minipage}\hspace{1.5cm}
\noindent
\begin{minipage}[t]{0.3\linewidth}
\begin{small}
\begin{verbatim}
\- (move margem pra esquerda)
\' (move pra coluna anterior)
\` (move para margem direita)
\kill (ignora linha)
\end{verbatim}
\end{small}
\end{minipage}

Veja alguns exemplos de construção de tabelas com o ambiente \texttt{\{tabular\}} (veja também pelo código-fonte deste tutorial). É possível traçar linhas verticais e horizontais, perfazendo um contorno. As linhas verticais são definidas com o caractere \texttt{|}\index{\textbar} entre as letras das colunas. As linhas horizontais são feitas pelo comando \texttt{\textbackslash hline} ou \texttt{\textbackslash cline\{\}}.

Uso do comando \texttt{\textbackslash cline\{i-f\}}:
\index{cline@\textbackslash cline}

\begin{center}
	\scalebox{0.90} {
	\begin{tabular}{ccccccc}
		  &   &   &   & 2 & 8 & 6 \\
		  &   & \texttimes &   & 8 & 2 & 6 \\
		\cline{4-7}
		  &   &   & 1 & 7 & 1 & 6 \\
		+ &   &   & 5 & 7 & 2 & \\
		  & 2 & 2 & 8 & 8 \\
		\cline{2-7}
		  & 2 & 3 & 6 & 2 & 3 & 6
	\end{tabular}
    }
\end{center}

\newpage
Apenas com o uso do caractere \texttt{|}:
\index{\textbar}

\begin{center}
	\begin{tabular}{|l|c|c|c|}
		& \textbf{Grande} & \textbf{Média} & \textbf{Pequena}\\
		Panela     & 5     & 0     & 3\\
		Frigideira & 2     & 3     & 3\\
		Chaleira   & 2     & 5     & 1\\
		Caçarola   & 7     & 1     & 0\\
		Leiteira   & 4     & 1     & 3\\
		Assadeira  & 4     & 4     & 0
	\end{tabular}
\end{center}

Outro recurso é esticar uma célula por mais de uma coluna, com o comando \texttt{\textbackslash multicolumn\,\{ncolunas\}\,\{alinhamento\}\,\{\textit{conteúdo}\}}:
\index{multicolumn@\textbackslash multicolumn}

\begin{samepage}
	\begin{center}
		\begin{tabular}{|ccc|ccc|}
			\hline
			\multicolumn{3}{|c|}{\cellcolor{blue!20}\textbf{Combinações dos bonés}} & \multicolumn{3}{c|}{\cellcolor{green!20}\textbf{Saberia a cor}} \\
			\cellcolor{blue!20}\textbf{Frente}   & \cellcolor{blue!20}\textbf{Meio}     & \cellcolor{blue!20}\textbf{Último}     & \cellcolor{green!20}\textbf{Frente} & \cellcolor{green!20}\textbf{Meio}   & \cellcolor{green!20}\textbf{Último} \\
			\hline
			Azul     & Azul     & Azul       & Sim    & Não    & Não \\
			Azul     & Azul     & Amarelo    & Sim    & Não    & Não \\
			Azul     & Amarelo  & Azul       & Sim    & Não    & Não \\
			Amarelo  & Azul     & Azul       & ---    & Sim    & Não \\
			Azul     & Amarelo  & Amarelo    & Sim    & Não    & Não \\
			Amarelo  & Azul     & Amarelo    & ---    & Sim    & Não \\
			Amarelo  & Amarelo  & Azul       & ---    & ---    & Sim \\
			\hline
		\end{tabular}
	\end{center}
\end{samepage}

Ou esticar uma célula por mais de uma linha, com o comando \texttt{\textbackslash multirow\,\{nlinhas\} \{largura\}\,\{\textit{conteúdo}\}}:
\index{multirow@\textbackslash multirow}

\begin{center}
\begin{tabular}{|l|r|l|}
	\hline
	\rowcolor{yellow!40}
	\multicolumn{3}{|c|}{\textbf{Seleção 70}} \\
	\hline
	Goleiro & 1 & Félix \\ \hline
	\multirow{4}{*}{Defesa} & 4 & Carlos Alberto \\
	& 2 & Brito \\
	& 3 & Piazza \\
	& 16 & Everaldo \\ \hline
	\multirow{3}{*}{Meias} & 5 & Clodoaldo \\
	& 8 & Gérson \\
	& 11 & Rivellino \\
\hline
	\multirow{3}{*}{Ataque} & 7 & Jairzinho \\
	& 9 & Tostão \\
	& 10 & Pelé \\
	\hline
\end{tabular}
\end{center}

\newpage
Também é possível colorir as tabelas, colorindo linhas com o comando \texttt{\textbackslash rowcolor\{\}}, colorindo colunas com o comando \texttt{\textbackslash columncolor\{\}}\index{columncolor@\textbackslash columncolor} e colorindo células com o comando \texttt{\textbackslash cellcolor\{\}}:\index{cellcolor@\textbackslash cellcolor}

%\noindent
\begin{minipage}[t]{0.4\linewidth}
\begin{footnotesize}
\begin{verbatim}
\begin{tabular}{|llll|}
\hline
\rowcolor{cyan!30}
11 & 12 & 13 & 14\\
\rowcolor{magenta!30}
21 & 22 & 23 & 24\\
\rowcolor{yellow!30}
31 & 32 & 33 & 34\\
\rowcolor{black!30}
41 & 42 & 43 & 44\\
\hline
\end{tabular}
\end{verbatim}
\end{footnotesize}

\begin{tabular}{|llll|}
\hline
\rowcolor{cyan!30}
11 & 12 & 13 & 14\\
\rowcolor{magenta!30}
21 & 22 & 23 & 24\\
\rowcolor{yellow!30}
31 & 32 & 33 & 34\\
\rowcolor{black!30}
41 & 42 & 43 & 44\\
\hline
\end{tabular}
\end{minipage}\hspace{\fill}
\begin{minipage}[t]{0.6\linewidth}
\begin{footnotesize}
\begin{verbatim}
\begin{tabular}{|>{\columncolor{cyan!30}}l
                 >{\columncolor{magenta!30}}l
                 >{\columncolor{yellow!30}}l
                 >{\columncolor{black!30}}l|}
\hline
11 & 12 & 13 & 14\\
21 & 22 & 23 & 24\\
31 & 32 & 33 & 34\\
41 & 42 & 43 & 44\\
\hline
\end{tabular}
\end{verbatim}
\end{footnotesize}

\begin{tabular}{|>{\columncolor{cyan!30}}l>{\columncolor{magenta!30}}l>{\columncolor{yellow!30}}l>{\columncolor{black!30}}l|}
\hline
11 & 12 & 13 & 14\\
21 & 22 & 23 & 24\\
31 & 32 & 33 & 34\\
41 & 42 & 43 & 44\\
\hline
\end{tabular}
\end{minipage}

Uma especificação por toda coluna pode ser aplicada no argumento do ambiente \texttt{\{tabular\}}, usando \texttt{>\{\textbackslash comando\}}\index{<\{\}} para comandos executados antes de cada elemento da coluna e \texttt{<\{\textbackslash comando\}}\index{>\{\}} para comandos que serão executados após cada elemento da coluna, como foi usado acima com o comando \texttt{\textbackslash columncolor\{\}}.

\subsection*{Lista}
\addcontentsline{toc}{subsection}{Lista}

As listas são construídas com o ambiente \texttt{\{itemize\}}\index{itemize}, para listas não-ordenadas, ou com o ambiente \texttt{\{enumerate\}}\index{enumerate}, para listas ordenadas:
\index{item@\textbackslash item}
\begin{multicols}{2}
\begin{verbatim}
\begin{itemize}
    \item Arroz;
    \item Feijão;
    \item Carne.
\end{itemize}
\end{verbatim}

\begin{itemize}
	\setlength\itemsep{0em}
	\item Arroz;
	\item Feijão;
	\item Carne.
\end{itemize}
\columnbreak
\begin{verbatim}
\begin{enumerate}
    \item Arroz;
    \item Feijão;
    \item Carne.
\end{enumerate}
\end{verbatim}

\begin{enumerate}
	\setlength\itemsep{0em}
	\item Arroz;
	\item Feijão;
	\item Carne.
\end{enumerate}
\end{multicols}

\newpage
O comando do ambiente \texttt{\{itemize\}} aceita especificar um caractere para indicar os itens da lista e com o ambiente \texttt{\{enumerate\}} também podemos definir um outro caractere de ordenação:
\index{label}
\begin{multicols}{2}
\begin{small}
\begin{verbatim}
\begin{itemize}[label=\ding{71}]
\end{verbatim}
\end{small}

\begin{itemize}[label=\ding{71}]
	\setlength\itemsep{0em}
	\item Arroz;
	\item Feijão;
	\item Carne.
\end{itemize}

\begin{small}
\begin{verbatim}
\begin{itemize}[label=$\rightarrow$]
\end{verbatim}
\end{small}

\begin{itemize}[label=$\rightarrow$]
	\setlength\itemsep{0em}
	\item Arroz;
	\item Feijão;
	\item Carne.
\end{itemize}

\begin{small}
\begin{verbatim}
\begin{itemize}[label=\ding{43}]
\end{verbatim}
\end{small}

\begin{itemize}[label=\ding{43}]
	\setlength\itemsep{0em}
	\item Arroz;
	\item Feijão;
	\item Carne.
\end{itemize}
\columnbreak
\begin{small}
\begin{verbatim}
\begin{enumerate}[label=\alph*.]
\end{verbatim}
\end{small}

%\begin{enumerate}[a)]
\begin{enumerate}[label=\alph*.]
	\setlength\itemsep{0em}
	\item Arroz;
	\item Feijão;
	\item Carne.
\end{enumerate}

\begin{small}
\begin{verbatim}
\begin{enumerate}[label=(\roman*)]
\end{verbatim}
\end{small}

\begin{enumerate}[label=(\roman*)]
	\setlength\itemsep{0em}
	\item Arroz;
	\item Feijão;
	\item Carne.
\end{enumerate}

\begin{small}
\begin{verbatim}
\begin{enumerate}[label=\Alph*)]
\end{verbatim}
\end{small}

\begin{enumerate}[label=\Alph*)]
	\setlength\itemsep{0em}
	\item Arroz;
	\item Feijão;
	\item Carne.
\end{enumerate}
\end{multicols}

Ainda é possível inserir lista dentro de item de lista, o \LaTeX\ altera automaticamente o símbolo de indicação de item.

Outro recurso é a lista dentro de um parágrafo, na mesma linha, por exemplo:
\begin{inparaenum}[a)]
	\item Primeiro
	\item Segundo
	\item Terceiro
\end{inparaenum}

Que foi produzida usando este comando:
\index{inparaenum}

\begin{verbatim}
   \begin{inparaenum}[a)]
      \item Primeiro
      \item Segundo
      \item Terceiro
   \end{inparaenum}
\end{verbatim}

Um item pode ter também um rótulo individual, com seu comando e o argumento para o rótulo: \texttt{\textbackslash item[rótulo]}.
\index{item@\textbackslash item}

\newpage
\subsection*{Verso}
\addcontentsline{toc}{subsection}{Verso}

A formatação do texto para compor a estrutura de um verso é com o bloco de ambiente \texttt{\{verse\}}\index{verse}. Usa-se também a quebra de linha tradicional (\texttt{\textbackslash\textbackslash}) para uma simples mudança de linha, contudo, existem alguns comandos para lidar com mais eficiência em relação as quebras, por exemplo \texttt{\textbackslash\textbackslash>[\textbackslash versewidth]}.
\index{versewidth@\textbackslash versewidth}

\begin{verse}
	\onehalfspacing
	Duas avós com suas duas netas.\\
	Dois maridos com suas duas esposas.\\
	Dois pais com suas duas filhas.\\
	Duas mães com seus dois filhos.\\
	Duas solteiras com suas mães.\\
	Duas irmãs com seus dois irmãos.\\
	Leia meus dizeres mas de todo esse pessoal dito,\\
	só falei de 6 pessoas, o que parece um mito!\\
	Ninguém nasceu proscrito, com incesto ou com delito.\\
	Não sei porque eu me agito e fico aflito.\\
	Quem são eles eu peço a algum perito!
\end{verse}

\subsection*{Nota de rodapé}
\addcontentsline{toc}{subsection}{Nota de rodapé}

Para inserir uma nota de rodapé\footnote{Uma anotação colocada ao pé de uma página.\label{notarodape}}, use o comando \texttt{\textbackslash footnote\{\textit{nota}\}}\index{footnote@\textbackslash footnote} logo após a palavra que será comentada.

\subsection*{Colunas}
\addcontentsline{toc}{subsection}{Colunas}

Existem diversas soluções para formatar o texto em colunas. Pode-se usar os comandos de ambiente:

\index{multicols@\textbackslash multicols}
\index{vwcol@\textbackslash vwcol}
\index{minipage@\textbackslash minipage}
\index{tabular@\textbackslash tabular}

\texttt{\small\textbackslash begin\{multicols\} ... \textbackslash end\{multicols\}}\\
\indent\texttt{\small\textbackslash begin\{vwcol\} ... \textbackslash end\{vwcol\}}\\
\indent\texttt{\small\textbackslash begin\{minipage\} ... \textbackslash end\{minipage\}}\\
\indent\texttt{\small\textbackslash begin\{tabular\} ... \textbackslash end\{tabular\}}\\

Os ambientes \texttt{\{multicols\}} e \texttt{\{vwcol\}} permitem uma certa fluidez do texto entre as colunas, uma mudança automática dos parágrafos entre as colunas. Sobre o ambiente \texttt{\{multicols\}}, o padrão é acontecer um equilíbrio na quantidade de texto em cada lado. Este balanço automático pode ser desligado usando o ambiente  \texttt{\{multicols*\}}.

Veja alguns exemplos. Nos ambientes de texto fluido foi usado o comando \texttt{\textbackslash columnbreak}\index{columnbreak@\textbackslash columnbreak} ou o comando \texttt{\textbackslash newpage}\index{newpage@\textbackslash newpage} para forçar a quebra de coluna:

\begin{small}\index{multicols@\textbackslash multicols}
\begin{verbatim}
   \begin{multicols}{2}
      \lipsum[66]\par
      \columnbreak
      \lipsum[75]
   \end{multicols}
\end{verbatim}
\end{small}

\begin{multicols}{2}
    \lipsum[66]\par
    \columnbreak
    \lipsum[75]
\end{multicols}

\begin{small}\index{minipage@\textbackslash minipage}
\begin{verbatim}
   \begin{minipage}[t]{0.55\linewidth}
      \setlength{\parindent}{1.5em}
      \lipsum[66]
   \end{minipage}\hspace{\fill}
   \begin{minipage}[t]{0.4\linewidth}
      \setlength{\parindent}{1.5em}
      \lipsum[75]
   \end{minipage}
\end{verbatim}
\end{small}

\noindent
\begin{minipage}[t]{0.55\linewidth}
	\setlength{\parindent}{1.5em}
	\lipsum[66]
\end{minipage}\hspace{\fill}
\begin{minipage}[t]{0.4\linewidth}
	\setlength{\parindent}{1.5em}
	\lipsum[75]
\end{minipage}

\newpage
\begin{small}\index{tabular@\textbackslash tabular}\index{parbox@\textbackslash parbox}
\begin{verbatim}
   \begin{tabular}{p{0.6\linewidth}p{0.4\linewidth}}
      \parbox{0.6\textwidth}{
         \setlength{\parindent}{1.5em}
         \lipsum[66]
      }
      &
      \parbox{0.4\textwidth}{
         \setlength{\parindent}{1.5em}
         \lipsum[75]
      }
   \end{tabular}
\end{verbatim}
\end{small}

\noindent
\begin{tabular}{p{0.6\linewidth}p{0.4\linewidth}}
	\parbox{0.6\textwidth}{
		\setlength{\parindent}{1.5em}
		\lipsum[66]
	}
	&
	\parbox{0.4\textwidth}{
		\setlength{\parindent}{1.5em}
		\lipsum[75]
	}
\end{tabular}

\begin{small}\index{vwcol@\textbackslash vwcol}
\begin{verbatim}
   \begin{vwcol}[widths={0.6,0.4},sep=1.5em,justify=flush,rule=0pt,indent=1.5em]
      \indent\lipsum[66]
      \newpage
      \lipsum[75]
   \end{vwcol}
\end{verbatim}
\end{small}

\noindent
\begin{vwcol}[widths={0.6,0.4},
              sep=1.5em,
              justify=flush,
              rule=0pt,
              indent=1.5em]
    \indent\lipsum[66]
	\newpage
	\parbox{0.4\textwidth}{
    \setlength{\parindent}{1.5em}
	\lipsum[75]
    }
\end{vwcol}

\chapter*{Índice, referências e ligações}
\addcontentsline{toc}{chapter}{Índice, referências e ligações}

\section*{Sumário}
\addcontentsline{toc}{section}{Sumário}

O sumário é a enumeração das principais divisões, capítulos, seções etc., seguindo a mesma ordem em que aparecem numa obra ou documento, geralmente com a indicação do número da página em que estas divisões se encontram. É também chamado de tabela de conteúdo e fica na parte frontal do livro.

No \LaTeX, a página do sumário é simplesmente criada automaticamente pelo comando \texttt{\textbackslash tableofcontents}\index{tableofcontents@\textbackslash tableofcontents}, colocado no início do ambiente \texttt{\{document\}}, logo após a página do título e da folha de rosto, se houver, e dentro parte marcada com \texttt{\textbackslash frontmatter}, se também houver este comando.

Caso não esteja usando um pacote de localização, é possível alterar o nome do sumário redefinindo o comando \texttt{\textbackslash contentsname}\index{contentsname@\textbackslash contentsname}, assim:

\texttt{\small\textbackslash renewcommand*\textbackslash contentsname\{Sumário\}}

Todo capítulo, seção e subseção, assim como os novos que vierem à ser adicionados no documento, serão incluídos automaticamente no sumário. Simples assim.

Caso não queira adicionar os capítulos, seções e subseções automaticamente, adicione o caractere \texttt{*} nos respectivos comandos. E para incluir manualmente estes capítulos, seções e subseções no sumário, use, após os comandos deles, o comando \texttt{\textbackslash addcontentsline}\index{addcontentsline@\textbackslash addcontentsline}. Exemplos para algumas divisões:
\begin{small}
\begin{verbatim}
    \chapter*{Nome do Capítulo}
    \addcontentsline{toc}{chapter}{Nome do Capítulo}

    \section*{Nome da Seção}
    \addcontentsline{toc}{section}{Nome da Seção}

    \subsection*{Nome da Subseção}
    \addcontentsline{toc}{subsection}{Nome da Subseção}
\end{verbatim}
\end{small}

\section*{Índice}
\addcontentsline{toc}{section}{Índice}

O índice, também chamado de índice analítico ou índice remissivo, é uma lista dos nomes ou termos mais importantes, em ordem alfabética, no fim de uma obra, com indicação da respectiva página em cada item.

No \LaTeX, o índice pode ser criado com os comandos do pacote \texttt{\{imakeidx\}}\index{imakeidx}, então, coloque no preâmbulo o comando: \texttt{\small\textbackslash usepackage\{imakeidx\}}

Ainda no preâmbulo, coloque o comando abaixo para disparar a criação do índice. Estes parâmetros são autoexplicativos:\index{makeindex@\textbackslash makeindex}

\texttt{\small\textbackslash makeindex[columns=3, title=Índice, intoc]}

Para cada termo no documento, que desejar incluir no índice, use próximo ao termo o comando \texttt{\textbackslash index\{termo\}}\index{index@\textbackslash index}. É possível incluir um apelido alternativo, específico para o critério de ordenação, usando a sintaxe \texttt{\{apelido@termo\}}.

Por fim, para criar a página do índice, coloque na parte traseira do documento o comando \texttt{\small\textbackslash printindex}\index{printindex@\textbackslash printindex}.

\section*{Referências}
\addcontentsline{toc}{section}{Referências}

No \LaTeX, quase tudo que está numerado pode ser referenciado e o \LaTeX\ automaticamente atualiza as referências, se houver alguma mudança posterior. Os objetos que podem ser referenciados são os capítulos, seções, subseções, equações, teoremas, notas de rodapé, figuras e tabelas. Os comandos, para tudo isso, são:

\index{label@\textbackslash label}
\index{ref@\textbackslash ref}
\index{pageref@\textbackslash pageref}

\begin{tabular}{lp{10cm}}
	\texttt{\small\textbackslash label\{marcador\}}
	& Usado para marcar um objeto, um identificador que será usado depois, na referência.\\
	\texttt{\small\textbackslash ref\{marcador\}}
	& Usado para referenciar um objeto com a respectiva marcação.\\
	\texttt{\small\textbackslash pageref\{marcador\}}
	& Usado para imprimir o número da página onde está o objeto com a respectiva marcação.\\
\end{tabular}

Saiba que, há uma convenção em adotar prefixos nas marcações:

\begin{tabular}{>{\ttfamily\small}ll>{\ttfamily\small}ll>{\ttfamily\small}ll}
ch:     & capítulo & tab:    & tabela  & fig:    & figura \\
sec:    & seção    & eq:     & equação & itm:    & item numerado \\
subsec: & subseção & lst:    & lista de código \\
\end{tabular}

Por exemplo: \texttt{\textbackslash label\{ch:introducao\}}. O que depois, será referenciado com: \texttt{\textbackslash ref\{ch:introducao\}}.

E aqui tem uma referência à nota de rodapé \ref{notarodape} do capítulo sobre formatação de texto, na página \pageref{notarodape} (veja esta parte no código-fonte).

\section*{Ligações internas}
\addcontentsline{toc}{section}{Ligações internas}

Semelhante à uma página em HTML, o \LaTeX\ também permite \textit{hyperlinks} dentro do documento. Podem ser ligações internas, para elementos do mesmo documento, como também ligações para arquivos externos ou endereços da Internet.

O uso do pacote \texttt{\{hyperref\}}\index{hyperref} transforma automaticamente todas as referências internas em ligações. Mas é possível adicionar trechos de textos que serão ligações para as marcações existentes, independente das referências.

Aproveitando o rótulo da marcação (\texttt{\textbackslash label\{marcador\}})\index{label@\textbackslash label}, para criar uma ligação interna, usa-se o comando \texttt{\textbackslash hyperref[marcador]\{texto\}}.\index{hyperref@\textbackslash hyperref}

Por exemplo, esta é uma ligação para a \hyperref[notarodape]{nota de rodapé} do capítulo sobre formatação de texto (veja esta parte no código-fonte).

O pacote \texttt{\{hyperref\}} possui uma extensa possibilidade de configuração. Veja o código-fonte deste tutorial para ver o comando \texttt{\textbackslash hypersetup\{\}}\index{hypersetup@\textbackslash hypersetup}, logo no preâmbulo. Este pacote também formata muitas coisas para o PDF exportado.

\section*{Ligações web}
\addcontentsline{toc}{section}{Ligações web}

As ligações para páginas Web são criadas com os comandos \texttt{\textbackslash href\{\}}\index{href@\textbackslash href} ou \texttt{\textbackslash url\{\}}\index{url@\textbackslash url}. a diferença entre eles é que \texttt{\textbackslash href} permite uma legenda para o endereço, enquanto o \texttt{\textbackslash url} imprime diretamente o endereço. Segue os exemplos:

\texttt{\small\textbackslash href\{https://www.ctan.org\}\{Comprehensive TeX Archive Network\}}

Resulta em toda esta expressão \href{https://www.ctan.org}{Comprehensive TeX Archive Network} com a ligação para a Web.

\texttt{\small\textbackslash url\{https://www.ctan.org\}}

Resulta em o próprio endereço \url{https://www.ctan.org} impresso e com a ligação para a Web.

\chapter*{Datas}
\addcontentsline{toc}{chapter}{Datas}

\section*{Comandos nativos}
\addcontentsline{toc}{section}{Comandos nativos}

Com os comandos nativos, provenientes desde o \TeX, pode-se obter os registros do ano, mês, dia e hora atual, usando \texttt{\textbackslash year}\index{year@\textbackslash year}, \texttt{\textbackslash month}\index{month@\textbackslash month}, \texttt{\textbackslash day}\index{day@\textbackslash day} e \texttt{\textbackslash time}\index{time@\textbackslash time} (este retorna os minutos desde a zero hora). Estes registros contém os valores coletados no momento do processamento do código-fonte .tex. Contudo, estes valores não são acessíveis com o uso direto destes comandos, sendo necessário utilizar em conjunto com outro comando nativo, que pode ser \texttt{\textbackslash the}\index{the@\textbackslash the} ou \texttt{\textbackslash number}.\index{number@\textbackslash number} 

Apenas esclarecendo, o comando \texttt{\textbackslash the} faz a conversão do valor do registro em uma string. Já o comando \texttt{\textbackslash number}, semelhante ao \texttt{\textbackslash the}, converte o valor do registro em um número. Para números inteiros, ambos podem ser usados.

Assim, os valores numéricos da data e hora podem ser impressos com os comandos:
\begin{small}
\begin{verbatim}
    \number\day
    \number\month
    \number\year
    \number\time
\end{verbatim}
\end{small}

Por exemplo, os comandos \texttt{\textbackslash number\textbackslash day/\textbackslash number\textbackslash month/\textbackslash number\textbackslash year}, nesta combinação, produz: \number\day/\number\month/\number\year.

\section*{Pacote datetime2}
\addcontentsline{toc}{section}{Pacote datetime2}

Na intenção de uma impressão da data e hora em um formato mais sofisticado, é necessário o uso de um pacote específico. O pacote \texttt{\{datetime2\}}\index{datetime2} é capaz de formatar a data por extenso, exibir o dia da semana e o mês pelo nome ao invés de números, formatar de acordo com a localização e idioma etc. Para usar este pacote, na língua portuguesa, inclua no preâmbulo:
\begin{small}
\begin{verbatim}
   \usepackage[portuges]{datetime2}
\end{verbatim}
\end{small}

Pode-se aproveitar a especificação de localização, caso já esteja definida, por exemplo pelo pacote \texttt{\{babel\}} ou na classe do documento, utilizando o parâmetro \texttt{[useregional]} no argumento do pacote \texttt{\{datetime2\}}. Ficando assim, no preâmbulo:
\begin{small}
\begin{verbatim}
   \usepackage[useregional]{datetime2}
\end{verbatim}
\end{small}

Mas, atenção com os pacotes de internacionalização e localização, como o pacote \texttt{\{babel\}}, pois, se carregado após, com outra localização, poderá ocasionar um conflito. Se precisar, consulte a documentação do pacote \texttt{\{datetime2\}}, lá possui algumas soluções para problemas deste tipo.

\subsection*{Data}
\addcontentsline{toc}{subsection}{Data}

Uma data específica pode ser exibida com os comandos:

\index{DTMdisplaydate@\textbackslash DTMdisplaydate}
\index{DTMdate@\textbackslash DTMdate}

\texttt{\small\textbackslash DTMdisplaydate\{AAAA\}\{MM\}\{DD\}\{DiaSemana\}} ou \texttt{\small\textbackslash DTMdate\{AAAA-MM-DD\}}

Obs.: O dia da semana precisa ser de -1(desativa), 0 (domingo) a 6 (sábado).

Desta forma, o comando \texttt{\textbackslash DTMdisplaydate\{1988\}\{10\}\{05\}\{-1\}} produz: \DTMdisplaydate{1988}{10}{05}{-1}. E o comando \texttt{\textbackslash DTMdate\{1988-10-05\}} produz: \DTMdate{1988-10-05}

Para imprimir a data atual, use o seguinte comando:
\index{DTMtoday@\textbackslash DTMtoday}

\texttt{\small\textbackslash DTMtoday}

Que produz algo como: \DTMtoday

Obs.: Existe o comando \texttt{\textbackslash today}\index{today@\textbackslash today}, suportado pelo pacote \texttt{\{datetime2\}}, porém é usado também por outros pacotes e então, para evitar imprevistos, prefira usar o \texttt{\textbackslash DTMtoday}, que é exclusivo.

Para imprimir o nome de um mês, na língua portuguesa, use o comando:
\index{DTMportugesmonthname@\textbackslash DTMportugesmonthname}

\texttt{\small\textbackslash DTMportugesmonthname\{n\}}

Assim, o comando \texttt{\textbackslash DTMportugesmonthname\{\textbackslash the\textbackslash month\}} produz: \DTMportugesmonthname{\the\month}

Para imprimir o nome de um dia da semana, na língua portuguesa, com letra minúscula ou maiúscula, use um dos comandos:

\index{DTMportugesweekdayname@\textbackslash DTMportugesweekdayname}
\index{DTMportugesWeekdayname@\textbackslash DTMportugesWeekdayname}

\texttt{\small\textbackslash DTMportugesweekdayname\{n\}} ou \texttt{\small\textbackslash DTMportugesWeekdayname\{n\}}

Por exemplo, cada um deles com o número 2 no argumento, produz: \DTMportugesweekdayname{2} e \DTMportugesWeekdayname{2}, respectivamente.

\subsection*{Hora}
\addcontentsline{toc}{subsection}{Hora}

Uma hora específica pode ser exibida com os comandos:

\index{DTMdisplaytime@\textbackslash DTMdisplaytime}
\index{DTMtime@\textbackslash DTMtime}

\texttt{\small\textbackslash DTMdisplaytime\{HH\}\{MM\}\{SS\}} ou \texttt{\small\textbackslash DTMtime\{HH:MM:SS\}}

Desta forma, o comando \texttt{\textbackslash DTMdisplaytime\{23\}\{18\}\{34\}} produz: \DTMdisplaytime{23}{18}{34}. E o comando \texttt{\textbackslash DTMtime\{23:18:34\}} produz: \DTMtime{23:18:34}

Para imprimir a hora atual, use o seguinte comando:

\index{DTMcurrenttime@\textbackslash DTMcurrenttime}

\texttt{\small\textbackslash DTMcurrenttime}

Que produz: \DTMcurrenttime

%\DTMcurrentzone

\subsection*{Data e hora}
\addcontentsline{toc}{subsection}{Data e hora}

Uma data e hora específica pode ser exibida com o comando:

\index{DTMdisplay@\textbackslash DTMdisplay}

\texttt{\small\textbackslash DTMdisplay\{AAAA\}\{MM\}\{DD\}\{DiaSemana\}\{HH\}\{MM\}\{SS\}\{HoraFuso\}\{MinutoFuso\}}

Assim, o comando \texttt{\textbackslash DTMdisplay\{2019\}\{09\}\{25\}\{-1\}\{18\}\{55\}\{37\}\{-03\}\{0\}}

produz: \DTMdisplay{2019}{09}{25}{-1}{18}{55}{37}{-03}{0}

A data e a hora atual pode ser exibida com o comando:

\index{DTMnow@\textbackslash DTMnow}

\texttt{\small\textbackslash DTMnow}

O qual produz: \DTMnow

\subsection*{Armazenando a data e hora}
\addcontentsline{toc}{subsection}{Armazenando a data e hora}

Ainda, existem comandos para armazenar uma data ou uma hora, com um nome escolhido e assim tornar possível o uso posterior no documento. Por exemplo, o comando \texttt{\textbackslash DTMsavedate\{minhadata\}\{2016-02-10\}}\index{DTMsavedate@\textbackslash DTMsavedate} armazena a data especificada em \texttt{minhadata} e o comando \texttt{\textbackslash DTMusedate\{minhadata\}}\index{DTMusedate@\textbackslash DTMusedate} imprime a data armazenada em \texttt{minhadata}.

A mesma coisa para a hora. Por exemplo, o comando \texttt{\textbackslash DTMsavetime\{minhahora\} \{10:00:00\}}\index{DTMsavetime@\textbackslash DTMsavetime} armazena a hora especificada em \texttt{minhahora} e o comando \texttt{\textbackslash DTMusetime \{minhahora\}}\index{DTMusetime@\textbackslash DTMusetime} imprime a hora armazenada em \texttt{minhahora}.

Para extrair os valores isoladamente, de uma data ou hora armazenada, existem os comandos: \texttt{\textbackslash DTMfetchyear\{nome\}}\index{DTMfetchyear@\textbackslash DTMfetchyear}, \texttt{\textbackslash DTMfetchmonth\{nome\}}\index{DTMfetchmonth@\textbackslash DTMfetchmonth}, \texttt{\textbackslash DTMfetchday\{nome\}}\index{DTMfetchday@\textbackslash DTMfetchday}, \texttt{\textbackslash DTMfetchhour\{nome\}}\index{DTMfetchhour@\textbackslash DTMfetchhour}, \texttt{\textbackslash DTMfetchminute\{nome\}}\index{DTMfetchminute@\textbackslash DTMfetchminute} e \texttt{\textbackslash DTMfetchsecond\{nome\}}\index{DTMfetchsecond@\textbackslash DTMfetchsecond}.

\subsection*{Estilos}
\addcontentsline{toc}{subsection}{Estilos}

Se preferir mudar o formato da data para valores numéricos, ao invés do formato por extenso, use o seguinte comando para especificar o formato preferido:

\index{DTMsetdatestyle@\textbackslash DTMsetdatestyle}

\texttt{\small\textbackslash DTMsetdatestyle\{nome\}}

Os nomes dos estilos são, resumidamente: \texttt{default}, \texttt{iso}, \texttt{ddmmyyyy}, \texttt{dmyyyy}, \texttt{dmyy}, \texttt{ddmmyy} ou \texttt{pdf}. 

\DTMsetdatestyle{ddmmyyyy}

Por exemplo, com o estilo \texttt{ddmmyyyy}, o comando \texttt{\textbackslash DTMtoday} produz: \DTMtoday

Se ainda, quiser uma data numérica mais regional, no português brasileiro, use os comandos abaixo:

\index{DTMsetup@\textbackslash DTMsetup}
\index{DTMtryregional@\textbackslash DTMtryregional}
\begin{small}
\begin{verbatim}
    \DTMsetup{useregional=numeric}
    \DTMtryregional{pt}{BR}
\end{verbatim}
\end{small}

\DTMsetup{useregional=numeric}
\DTMtryregional{pt}{BR}

Assim, o comando \texttt{\textbackslash DTMtoday} irá imprimir: \DTMtoday

\DTMsettimestyle{hmmss}

O formato da hora possui um estilo disponível, com o comando \texttt{\textbackslash DTMsettimestyle \{hmmss\}}\index{DTMsettimestyle@\textbackslash DTMsettimestyle} a hora impressa será no formato: \DTMtime{5:7:20}

%\DTMsetstyle{nome}

\chapter*{Matemática}
\addcontentsline{toc}{chapter}{Matemática}

O \TeX, de onde provém o \LaTeX, foi originalmente desenvolvido para facilitar a tipografia matemática, sendo capaz de formatar as mais variadas fórmulas e equações matemáticas. Assim, os recursos para este tipo de conteúdo são vastos, o grau de precisão é bastante alto.

\section*{Em linha}
\addcontentsline{toc}{section}{Em linha}

Para imprimir uma expressão matemática no parágrafo, na mesma linha, pode-se usar os delimitadores de ambiente \texttt{\$} e \texttt{\$}\index{\$}, \texttt{\textbackslash (}\index{\textbackslash (} e \texttt{\textbackslash )}\index{\textbackslash )} ou \texttt{\textbackslash begin\{math\}} e \texttt{\textbackslash end\{math\}}\index{math}. Por exemplo, \( ax^2 + bx + c = 0 \)  pode ser impresso digitando \texttt{\textbackslash ( ax\textasciicircum2 + bx + c = 0 \textbackslash )}

\section*{No modo de exibição}
\addcontentsline{toc}{section}{No modo de exibição}

\[x_{1,2}=\frac{-b\pm\sqrt{b^2-4ac}}{2a}\]

Para imprimir uma expressão em uma nova linha, chamado modo de exibição, usa-se os delimitadores \texttt{\textbackslash [}\index{\textbackslash [} e \texttt{\textbackslash ]}\index{\textbackslash ]} ou \texttt{\textbackslash begin\{displaymath\}} e \texttt{\textbackslash end\{displaymath\}}\index{displaymath}. Por exemplo, \texttt{\textbackslash [ \textbackslash frac\{9\}\{12\} + \textbackslash frac\{5\}\{34\} + \textbackslash frac\{7\}\{68\} = 1 \textbackslash ]} produz:

\[\frac{9}{12} + \frac{5}{34} + \frac{7}{68} = 1\]

Uma outra forma de imprimir uma expressão matemática no modo de exibição, inclusive alinhá-la, numerá-la e indexá-la no documento, é com os ambientes \texttt{\{equation\}}\index{equation} ou \texttt{\{align\}}\index{align}. Exemplos:
\begin{small}
\begin{verbatim}
   \begin{align}
      (2x \times 100 + x \times 10) - (x \times 100 + 2x \times 10) &= 270\\
      200x + 10x - 100x - 20x &= 270\\ 90x &= 270\\ x &= 270/90\\ x &= 3
   \end{align}
\end{verbatim}
\end{small}

Produz as equações numeradas:

\begin{align}
(2x \times 100 + x \times 10) - (x \times 100 + 2x \times 10) &= 270\\
200x + 10x - 100x - 20x &= 270\\
90x &= 270\\
x &= 270/90\\
x &= 3
\end{align}

\begin{small}\index{aligned}
\begin{verbatim}
   \begin{equation*}
      \begin{aligned}[c]
         x + y &= 90 && \text{\footnotesize{(1ª linha multiplica por -13)}}\\
         13x + 16y &= 1260 \\
         \\
         -13x - 13y &= -1170 \\
         13x + 16y &= 1260 \\
         \cline{1-2}
         3y &= 90 \\
         y &= 30
      \end{aligned}
      \begin{aligned}[c]
         \mathrm{Se:}\hspace{0.5cm} x + y &= 90\\
         \\
         \mathrm{ent\tilde{a}o:}\hspace{0.5cm} x + 30 &= 90\\
         x &= 60
      \end{aligned}
   \end{equation*}
\end{verbatim}
\end{small}

Produz todo este sistema de equações:

\begin{equation*}
\begin{aligned}[c]
x + y &= 90 && \text{\footnotesize{(1ª linha multiplica por -13)}}\\
13x + 16y &= 1260 \\
\\
-13x - 13y &= -1170 \\
13x + 16y &= 1260 \\
\cline{1-2}
3y &= 90 \\
y &= 30
\end{aligned}
\begin{aligned}[c]
\mathrm{Se:}\hspace{0.5cm} x + y &= 90\\
\\
\mathrm{ent\tilde{a}o:}\hspace{0.5cm} x + 30 &= 90\\
x &= 60
\end{aligned}
\end{equation*}

Obs.: O caractere \texttt{\&}\index{\&} marca a posição de alinhamento. Nestas sequencias de equações, foi adotada a posição no sinal de igual. O asterisco no comando cancela a numeração da equação no documento.

\newpage
\section*{Notações}
\addcontentsline{toc}{section}{Notações}

Veja alguns exemplos de notação de elementos matemáticos e a sintaxe dos comandos para cada um deles:

\index{\textbackslash \_}
\index{textasciicircum@\textbackslash textasciicircum}
\index{sqrt@\textbackslash sqrt}
\index{frac@\textbackslash frac}
\index{sum@\textbackslash sum}
\index{limits@\textbackslash limits}
\index{infty@\textbackslash infty}
\index{int@\textbackslash int}
\begin{center}
\rowcolors{1}{green!5}{pink!10}
\renewcommand{\arraystretch}{2} % define o espaço entre as linhas
\begin{tabular}{|m{9cm}|l|}
\hline
\texttt{\textbackslash( (x\_1,y\_2) \textbackslash)} & \( (x_1,y_2) \)\\
\texttt{\textbackslash( (x\textasciicircum2,y\textasciicircum2) \textbackslash)} & \( (x^2,y^2) \)\\
\texttt{\textbackslash( (x\_1\textasciicircum2,y\_2\textasciicircum3) \textbackslash)} & \( (x_1^2,y_2^3) \)\\
\texttt{\textbackslash( \textbackslash sqrt\{x\} \textbackslash)} & \( \sqrt{x} \)\\
\texttt{\textbackslash( \textbackslash sqrt[3]\{x\} \textbackslash)} & \( \sqrt[3]{x} \)\\
\texttt{\textbackslash( \textbackslash frac\{x\}\{y\} \textbackslash)} & \( \frac{x}{y} \)\\
\texttt{\textbackslash( \textbackslash sum\_\{n=1\}\textasciicircum\{3\}n \textbackslash)} & \( \sum_{n=1}^{3}n \)\\
\texttt{\textbackslash( \textbackslash sum\textbackslash limits\_\{n=1\}\textasciicircum\{n=3\}n \textbackslash)} & \( \sum\limits_{n=1}^{n=3}n \)\\[0.2cm]
\texttt{\textbackslash( \textbackslash lim\textbackslash limits\_\{n \textbackslash to \textbackslash infty\}n \textbackslash)} & \( \lim\limits_{n \to \infty}n \)\\[0.2cm]
\texttt{\textbackslash( \textbackslash int\_\{0\}\textasciicircum\{2\}x\textbackslash,dx \textbackslash)} & \( \int_{0}^{2}x\,dx \)\\
\hline
\end{tabular}
\end{center}

Alguns exemplos de expressões enormes:

\[
\sum_{j \in \mathbf{N}} b_{ij} \hat{y}_{j} =\sum_{j \in \mathbf{N}} b^{(\lambda)}_{ij} \hat{y}_{j} +(b_{ii} - \lambda_{i}) \hat{y}_{i} \hat{y}
\]

\[
\int_{\mathcal{D}} | \overline{\partial u} |^{2}\Phi_{0}(z) e^{\alpha |z|^2} \geq c_{4} \alpha \int_{\mathcal{D}} |u|^{2} \Phi_{0}e^{\alpha |z|^{2}} + c_{5} \delta^{-2} \int_{A} |u|^{2} \Phi_{0} e^{\alpha |z|^{2}}
\]

\[
\mathbf{A} =
\begin{pmatrix}
	\dfrac{\varphi \cdot X_{n, 1}}
	      {\varphi_{1} \times \varepsilon_{1}}
	& (x + \varepsilon_{2})^{2} & \cdots
	& (x + \varepsilon_{n - 1})^{n - 1}
	& (x + \varepsilon_{n})^{n}\\
	\dfrac{\varphi \cdot X_{n, 1}}
	      {\varphi_{2} \times \varepsilon_{1}}
	& \dfrac{\varphi \cdot X_{n, 2}}
	        {\varphi_{2} \times \varepsilon_{2}}
	& \cdots & (x + \varepsilon_{n - 1})^{n - 1}
	& (x + \varepsilon_{n})^{n}\\
	\hdotsfor{5}\\
	\dfrac{\varphi \cdot X_{n, 1}}
	      {\varphi_{n} \times \varepsilon_{1}}
	& \dfrac{\varphi \cdot X_{n, 2}}
	        {\varphi_{n} \times \varepsilon_{2}}
	& \cdots & \dfrac{\varphi \cdot X_{n,n-1}}
	                 {\varphi_{n} \times \varepsilon_{n - 1}}
	& \dfrac{\varphi\cdot X_{n, n}}
	        {\varphi_{n} \times \varepsilon_{n}}
\end{pmatrix}
+ \mathbf{I}_{n}
\]

\chapter*{Gráficos}
\addcontentsline{toc}{chapter}{Gráficos}

\section*{Imagens externas}
\addcontentsline{toc}{section}{Imagens externas}

Com o uso do pacote \texttt{\{graphicx\}}\index{graphicx} é fácil inserir imagens no documento em \LaTeX. Além de inserir, é possível posicionar, redimensionar e rotacionar.

\subsection*{Caminho}
\addcontentsline{toc}{subsection}{Caminho}

Antes, pode ser útil definir previamente o caminho das imagens que irão compor o documento. O comando \texttt{\textbackslash graphicspath\{\}} pode ser declarado no preâmbulo.\index{graphicspath@\textbackslash graphicspath}

A melhor maneira é especificar o caminho relativo às imagens. Pode ser relativo ao arquivo .tex que carrega a imagem ou pode ser relativo ao arquivo .tex principal, quando há mais de um arquivo .tex no projeto.

Um caminho relativo ao arquivo que carrega a imagem, segue o exemplo:

\texttt{\small\textbackslash graphicspath\{\{imagens/\}\}}

Um caminho relativo ao arquivo .tex principal, segue o exemplo:

\texttt{\small\textbackslash graphicspath\{\{./imagens/\}\}}

O caminho pode ser absoluto, quando há o caminho exato ao arquivo. Os exemplos para uma situação no MS Windows e no Linux são, respectivamente:

\texttt{\small\textbackslash graphicspath\{\{c:/usuario/imagens/\}\}}

\texttt{\small\textbackslash graphicspath\{\{/home/usuario/imagens/\}\}}

Pode ainda combinar múltiplos caminhos no mesmo comando, se as imagens estão em mais de uma pasta. Por exemplo:

\texttt{\small\textbackslash graphicspath\{\{./imagens1/\}\{./imagens2/\}\}}

\subsection*{Carregamento}
\addcontentsline{toc}{subsection}{Carregamento}

O carregamento da imagem é pelo comando \texttt{\textbackslash includegraphics[]\{\}}\index{includegraphics@\textbackslash includegraphics}. Se a imagem estiver no caminho, basta informar o nome do arquivo, por exemplo:

\texttt{\small\textbackslash includegraphics[scale=0.2]\{plataforma.jpg\}}

\begin{figure}[h!]
	\centering
    \includegraphics[scale=0.15]{plataforma.jpg}
    \caption{A plataforma de pesca.}
    \label{fig:plataforma}
\end{figure}

O argumento opcional recebe os parâmetros para escalonamento, dimensão, rotação etc. Veja os exemplos:

\begin{small}
\begin{verbatim}
   \includegraphics[width=3cm, height=4cm]{logo.png}
   \includegraphics[width=\textwidth]{lua.jpg}
   \includegraphics[scale=1.2, angle=45]{foto.jpg}
\end{verbatim}
\end{small}

Por recomendação, a extensão do arquivo pode ser omitida, assim o \LaTeX\ irá procurar por todos os formatos suportados de imagens.

\subsection*{Posicionamento}
\addcontentsline{toc}{subsection}{Posicionamento}

O carregamento da imagem torna-se mais preciso se estiver no ambiente \texttt{\{figure\}}\index{figure}. Com este ambiente podemos especificar o parâmetro do posicionamento:
\begin{small}
\begin{verbatim}
   \begin{figure}[h!]
      \centering
      \includegraphics[scale=0.2]{plataforma.jpg}
      \caption{A plataforma de pesca.}
      \label{fig:plataforma}
   \end{figure}
\end{verbatim}
\end{small}

\begin{tabular}{lp{10cm}}
\textbf{Parâmetro} & \textbf{Posição} \\
\texttt{h \footnotesize{(here)}} & Posição flutuante aqui mesmo. \\
\texttt{t \footnotesize{(top)}} & No topo da página. \\
\texttt{b \footnotesize{(bottom)}} & No pé da página. \\
\texttt{p \footnotesize{(page)}} & Coloca na página flutuante especial. \\
\texttt{!\ \footnotesize{(override)}} & Sobrepõe o cálculo do \LaTeX\ para a posição flutuante. \\
\end{tabular}

\newpage
Adicionando o pacote \texttt{\{wrapfig\}}\index{wrapfig}, o texto ganha a possibilidade de envolver a imagem carregada. Para isso usa-se o ambiente \texttt{\{wrapfigure\}}\index{wrapfigure}. Segue uma ilustração:
\begin{small}
\begin{verbatim}
   \begin{wrapfigure}{r}{0.35\textwidth}
      \centering
      \includegraphics[width=0.35\textwidth]{copacabana.jpg}
   \end{wrapfigure}
\end{verbatim}
\end{small}

Em um exemplo de envolvimento:

\begin{wrapfigure}{r}{0.35\textwidth}
	\centering
	\includegraphics[width=0.35\textwidth]{copacabana.jpg}
\end{wrapfigure}

\lipsum[1-2]

\section*{Plotagem de dados}
\addcontentsline{toc}{section}{Plotagem de dados}

Baseado no pacote PGF/TikZ existe o pacote \texttt{\{pgfplots\}}\index{pgfplots}, para construir uma plotagem de dados, de funções etc. Conta com um segundo componente, o pacote \texttt{\{pgfplotstable\}}\index{pgfplotstable}, que faz a formatação e o processamento de tabelas numéricas. Os ambientes de plotagem do \texttt{\{pgfplots\}} dependem do ambiente \texttt{\{tikzpicture\}}.\index{tikzpicture}

Provavelmente, o \texttt{\{pgfplots\}} é o pacote mais complexo do \LaTeX, seu manual tem quase 600 páginas, pois possui recursos tão poderosos quanto aos disponíveis nos melhores softwares matemáticos. Vale a pena a leitura do manual.

\subsection*{Configuração}
\addcontentsline{toc}{subsection}{Configuração}

Antes de executar as plotagens, é interessante configurar o \texttt{pgfplots} e isto é feito pelo comando \texttt{\textbackslash pgfplotsset}\index{pgfplotsset@\textbackslash pgfplotsset}, logo no preâmbulo ou diretamente no ambiente do gráfico.

Um primeiro ponto é quanto a compatibilidade do pacote. O \texttt{\{pgfplots\}} é desenvolvido tendo o cuidado com as versões anteriores, com os comandos que se tornaram obsoletos ou com os comandos que a sua versão instalada ainda não suporta. Por isso, inclua o parâmetro \texttt{compat=}\index{compat} mais o número da versão do pacote \texttt{\{pgfplots\}} que está instalado em seu sistema, ou uma versão anterior, se seu documento utiliza algum comando obsoleto.

O comando \texttt{\textbackslash pgfplotsset} também aceita os parâmetros de formatação do gráfico, como dimensão, estilos, fontes de caracteres etc.:

\texttt{\small\textbackslash pgfplotsset\{width=8cm, compat=1.16\}}

\subsection*{Ambientes}
\addcontentsline{toc}{subsection}{Ambientes}

Diversos ambientes irão estruturar uma plotagem, começando pelo ambiente gráfico \texttt{\{tikzpicture\}}\index{tikzpicture}, que forma a imagem. Interiormente, tem-se os ambientes específicos da plotagem, em relação aos eixos de escala normal ou escala logarítmica, com os ambientes \texttt{\{axis\}}\index{axis}, \texttt{\{semilogaxis\}}\index{semilogaxis} ou \texttt{\{loglogaxis\}}\index{loglogaxis}. Sintaxe básica:
\begin{small}
\begin{verbatim}
   \begin{tikzpicture}
      \begin{axis}[ ]
      ...
      \end{axis}
   \end{tikzpicture}
\end{verbatim}
\end{small}

Os ambientes dos eixos podem receber um argumento com os parâmetros para formatar cada eixo, separados por vírgulas e seguindo a boa prática de digitá-los um por linha, por exemplo:
\begin{small}
\begin{verbatim}
   \begin{axis}[
      title = Título,
      xlabel = {$x$},
      ylabel = {$y$},
   ]
\end{verbatim}
\end{small}

Dica: Externamente ao ambiente \texttt{\{tikzpicture\}} pode-se colocar o ambiente \texttt{\{figure\}}, que permite, por exemplo, adicionar uma legenda à figura e outros tratamentos.

\subsection*{Plotagem}
\addcontentsline{toc}{subsection}{Plotagem}

Dentro do ambiente do eixo insere-se o comando para adicionar uma plotagem, com \texttt{\textbackslash addplot}\index{addplot@\textbackslash addplot}, ou \texttt{\textbackslash addplot3}\index{addplot3@\textbackslash addplot3} para visualização em 3D. Neste comando, especifica-se a origem dos dados, se será por uma função, por coordenadas ou fornecidos por uma tabela em um arquivo externo. Também pode receber um argumento com diversos parâmetros:\index{coordinates}\index{table}
\begin{small}
\begin{verbatim}
   \addplot[
      blue,
      domain=-6:4,
   ]
   {x^2 + 2*x + 1};

   \addplot[
      red,
      mark=square,
   ]
   coordinates {(1,35)(2,34)(3,30)(4,26)(5,20)(6,17)};

   \addplot table {dados.txt};
\end{verbatim}
\end{small}

O comando \texttt{\textbackslash addplot} pode ser usado mais de uma vez no mesmo gráfico, caso tenha outros dados compatíveis ao mesmo domínio.

\subsection*{Exemplos de plotagem}
\addcontentsline{toc}{subsection}{Exemplos de plotagem}

\subsubsection*{Função matemática:}

\noindent
\begin{minipage}[t]{0.5\linewidth}
\begin{small}
\begin{verbatim}
\begin{tikzpicture}
    \begin{axis}[
        axis lines = left,
        title = {Equação do 2º grau},
        xlabel = {$x$},
        ylabel = {$f(x)$},
    ]
        \addplot[domain = -6:4,
                 samples = 20,
                 color = blue,]
        {x^2 + 2*x + 1};
        \addlegendentry{$x^2 + 2x + 1$}
    \end{axis}
\end{tikzpicture}
\end{verbatim}
\end{small}
\end{minipage}\hspace{\fill}
\begin{minipage}[t]{0.4\linewidth}
\strut\vspace*{-\baselineskip}\newline
\begin{tikzpicture}
\begin{axis}[
    axis lines = left,
    title = {Equação do 2º grau},
    xlabel = {$x$},
    ylabel = {$f(x)$},
]
\addplot[
    domain = -6:4,
    samples = 20,
    color = blue,
]
{x^2 + 2*x + 1};
\addlegendentry{$x^2 + 2x + 1$}
\end{axis}
\end{tikzpicture}
\end{minipage}

\subsubsection*{Coordenadas logarítmicas:}

\noindent
\begin{minipage}[t]{0.5\linewidth}
\begin{small}
\begin{verbatim}
\begin{tikzpicture}
    \begin{loglogaxis}[
        xlabel = {Graus de liberdade},
        ylabel = {$L_2$ Erro},]

    \addplot coordinates {
        (5,8.312e-02) (49,7.407e-03)
        (321,5.874e-04) (1793,4.442e-05)
        (9217,3.261e-06) };

    \addplot coordinates {
        (7,8.472e-02) (111,1.022e-02)
        (1023,1.039e-03) (7423,9.658e-05)
        (47103,8.437e-06) };

    \legend{$d=2$,$d=3$}
    \end{loglogaxis}
\end{tikzpicture}
\end{verbatim}
\end{small}
\end{minipage}\hspace{\fill}
\begin{minipage}[t]{0.4\linewidth}
\strut\vspace*{-\baselineskip}\newline
\begin{tikzpicture}
\begin{loglogaxis}[
    xlabel = {Graus de liberdade},
    ylabel = {$L_2$ Erro},]
                   
\addplot coordinates {
	(5,8.312e-02) (49,7.407e-03)
	(321,5.874e-04) (1793,4.442e-05)
	(9217,3.261e-06)
};

\addplot coordinates {
	(7,8.472e-02) (111,1.022e-02)
	(1023,1.039e-03) (7423,9.658e-05)
	(47103,8.437e-06)
};

\legend{$d=2$,$d=3$}
\end{loglogaxis}
\end{tikzpicture}
\end{minipage}

\subsubsection*{Coordenadas em escala normal:}

\noindent
\begin{minipage}[t]{0.5\linewidth}
\begin{small}
\begin{verbatim}
\begin{tikzpicture}
\begin{axis}[
    title = {Temperaturas},
    xlabel = {Meses},
    ylabel = {°C},
    xmin = 1, xmax = 12,
    ymin = 15, ymax = 40,
    xtick = {1,2,3,4,5,6,7,8,9,10,11,12},
    ytick = {0,15,20,25,30,35,40},
    ymajorgrids = true,
    grid style = dashed,]
\addplot[blue,
         mark = square]
coordinates {
    (1,35)(2,34)(3,30)(4,26)(5,20)(6,17)
    (7,15)(8,16)(9,20)(10,26)(11,30)(12,32)};
\end{axis}
\end{tikzpicture}
\end{verbatim}
\end{small}
\end{minipage}\hspace{\fill}
\begin{minipage}[t]{0.4\linewidth}
\strut\vspace*{-\baselineskip}\newline
\begin{tikzpicture}
\begin{axis}[
    title = {Temperaturas},
    xlabel = {Meses},
    ylabel = {°C},
    xmin = 1, xmax = 12,
    ymin = 15, ymax = 40,
    xtick = {1,2,3,4,5,6,7,8,9,10,11,12},
    ytick = {0,15,20,25,30,35,40},
    ymajorgrids = true,
    grid style = dashed,]
\addplot[blue,
         mark = square]
coordinates {
	(1,35)(2,34)(3,30)(4,26)(5,20)(6,17)
	(7,15)(8,16)(9,20)(10,26)(11,30)(12,32)};
\end{axis}
\end{tikzpicture}
\end{minipage}

\subsubsection*{Plotagem em barra:}

\noindent
\begin{minipage}[t]{0.5\linewidth}
\begin{small}
\begin{verbatim}
\begin{tikzpicture}
\begin{axis}[
    ybar,
    enlargelimits = 0.3,
    legend style = {at={(0.5,-0.2)},
        anchor = north,legend columns = -1},
    ylabel = {passageiros},
    symbolic x coords={2017,2018,2019},
    xtick = data,
    nodes near coords,
    nodes near coords align={vertical},]
\addplot coordinates {(2017,7) (2018,9) (2019,5)};
\addplot coordinates {(2017,4) (2018,6) (2019,4)};
\addplot coordinates {(2017,2) (2018,2) (2019,1)};
\legend{homens,mulheres,crianças}
\end{axis}
\end{tikzpicture}
\end{verbatim}
\end{small}
\end{minipage}\hspace{\fill}
\begin{minipage}[t]{0.4\linewidth}
\strut\vspace*{-\baselineskip}\newline
\begin{tikzpicture}
\begin{axis}[
	ybar,
	enlargelimits = 0.3,
	legend style = {at={(0.5,-0.2)},
		anchor = north,legend columns = -1},
	ylabel = {passageiros},
	symbolic x coords={2017,2018,2019},
	xtick = data,
	nodes near coords,
	nodes near coords align={vertical},
]
\addplot coordinates {(2017,7) (2018,9) (2019,5)};
\addplot coordinates {(2017,4) (2018,6) (2019,4)};
\addplot coordinates {(2017,2) (2018,2) (2019,1)};
\legend{homens,mulheres,crianças}
\end{axis}
\end{tikzpicture}
\end{minipage}

\subsubsection*{Plotagem 3D:}

A plotagem em 3 dimensões é com o comando \texttt{\textbackslash addplot3}. Veja dois interessantes exemplos com os parâmetros surf e mesh, que criam uma superfície e uma malha, respectivamente:

\noindent
\begin{minipage}[t]{0.5\linewidth}
\begin{small}
\begin{verbatim}
\begin{tikzpicture}
\begin{axis}[title = {Superfície},]
\addplot3[surf, domain = 0:360,
          samples = 40,]
{sin(x)*sin(y)};
\end{axis}
\end{tikzpicture}
\end{verbatim}
\end{small}
\begin{tikzpicture}
\begin{axis}[title={Superfície},]
\addplot3[surf,
          domain=0:360,
          samples=40,]
{sin(x)*sin(y)};
\end{axis}
\end{tikzpicture}
\end{minipage}\hspace{\fill}
\begin{minipage}[t]{0.5\linewidth}
\begin{small}
\begin{verbatim}
\begin{tikzpicture}
\begin{axis}[title = Malha, hide axis,
             colormap/cool,]
\addplot3[mesh, samples = 50,
          domain = -8:8,]
{sin(deg(sqrt(x^2+y^2)))/sqrt(x^2+y^2)};
\end{axis}
\end{tikzpicture}
\end{verbatim}
\end{small}
\begin{tikzpicture}
\begin{axis}[title=Malha,
             hide axis,
             colormap/cool,]
\addplot3[mesh,
          samples=50,
          domain=-8:8,]
{sin(deg(sqrt(x^2+y^2)))/sqrt(x^2+y^2)};
\end{axis}
\end{tikzpicture}
\end{minipage}

\chapter*{Caracteres e símbolos}
\addcontentsline{toc}{chapter}{Caracteres e símbolos}

Alguns caracteres acentuados e os caracteres de símbolos pedem um comando específico e, normalmente, o uso de um pacote.

\section*{Acentuação}
\addcontentsline{toc}{section}{Acentuação}

Os caracteres acentuados podem ser digitados diretamente no documento mas, em algumas situações, é necessário usar o comando equivalente para impressão.

\subsection*{Modo texto}
\addcontentsline{toc}{subsection}{Modo texto}

Os seguintes comandos devem ser utilizados nos parágrafos ou no modo esquerda-direita (LR) e, se omitida a letra, apenas o acento é impresso:

\noindent
\begin{minipage}[t]{0.5\linewidth}
\begin{itemize}[label={}]
\setlength\itemsep{-0.5em}
\item \texttt{\textbackslash `\{a\}} grave: \`{a}
\item \texttt{\textbackslash '\{e\}} agudo: \'{e}
\item \texttt{\textbackslash \^{}\{o\}} circunflexo: \^{o}
\item \texttt{\textbackslash "\{u\}} trema: \"{u}
\item \texttt{\textbackslash H\{o\}} trema húngaro longo: \H{o}
\item \texttt{\textbackslash \~{}\{a\}} til: \~{a}
\item \texttt{\textbackslash c\{c\}} cedilha: \c{c}
\end{itemize}
\end{minipage}\hspace{\fill}
\begin{minipage}[t]{0.5\linewidth}
\begin{itemize}[label={}]
\setlength\itemsep{-0.5em}
\item \texttt{\textbackslash =\{a\}} mácron (barra em cima): \={a}
\item \texttt{\textbackslash b\{o\}} barra embaixo: \b{o}
\item \texttt{\textbackslash .\{z\}} ponto em cima: \.{z}
\item \texttt{\textbackslash d\{o\}} ponto embaixo: \d{o}
\item \texttt{\textbackslash u\{e\}} braquia (breve): \u{e}
\item \texttt{\textbackslash v\{c\}} caron: \v{c}
\item \texttt{\textbackslash t\{oo\}} braquia invertida: \t{oo}
\end{itemize}
\end{minipage}

\subsection*{Modo matemático}
\addcontentsline{toc}{subsection}{Modo matemático}

Estes, somente são aceitos no modo matemático:

\index{grave@\textbackslash grave}
\index{acute@\textbackslash acute}
\index{hat@\textbackslash hat}
\index{widehat@\textbackslash widehat}
\index{tilde@\textbackslash tilde}
\index{widetilde@\textbackslash widetilde}
\index{mathring@\textbackslash mathring}
\index{bar@\textbackslash bar}
\index{dot@\textbackslash dot}
\index{ddot@\textbackslash ddot}
\index{breve@\textbackslash breve}
\index{check@\textbackslash check}
\index{vec@\textbackslash vec}

\noindent
\begin{minipage}[t]{0.5\linewidth}
\begin{itemize}[label={}]
\setlength\itemsep{-0.5em}
\item \texttt{\textbackslash grave\{a\}} grave: $\grave{a}$
\item \texttt{\textbackslash acute\{e\}} agudo:
$\acute{e}$
\item \texttt{\textbackslash hat\{o\}} circunflexo: $\hat{o}$
\item \texttt{\textbackslash widehat\{oo\}} circunflexo largo: $\widehat{oo}$
\item \texttt{\textbackslash tilde\{a\}} til:
$\tilde{a}$
\item \texttt{\textbackslash widetilde\{aa\}} til largo: $\widetilde{aa}$
\item \texttt{\textbackslash mathring\{a\}} anel: $\mathring{a}$
\end{itemize}
\end{minipage}\hspace{\fill}
\begin{minipage}[t]{0.5\linewidth}
\begin{itemize}[label={}]
\setlength\itemsep{-0.5em}
\item \texttt{\textbackslash bar\{a\}} mácron (barra em cima):
$\bar{a}$
\item \texttt{\textbackslash dot\{z\}} ponto em cima:
$\dot{z}$
\item \texttt{\textbackslash ddot\{u\}} trema:
$\ddot{u}$
\item \texttt{\textbackslash breve\{e\}} braquia (breve):
$\breve{e}$
\item \texttt{\textbackslash check\{c\}} caron:
$\check{c}$
\item \texttt{\textbackslash vec\{v\}} vetor:
$\vec{v}$
\end{itemize}
\end{minipage}

\section*{Caracteres}
\addcontentsline{toc}{section}{Caracteres}

Além dos caracteres especiais, existem os caracteres simbólicos que, alguns por não existirem no teclado, dependem de comandos específicos para serem impressos no documento.

\subsection*{Alfabeto grego}
\addcontentsline{toc}{subsection}{Alfabeto grego}

Estes comandos nativos do \LaTeX\ funcionam somente no modo matemático.

\noindent
\begin{tabular}{l>{\ttfamily\footnotesize}ll>{\ttfamily\footnotesize}ll>{\ttfamily\footnotesize}ll>{\ttfamily\footnotesize}ll>{\ttfamily\footnotesize}l}
$\alpha$
& \textbackslash alpha & 
$\beta$
& \textbackslash beta &
$\gamma$
& \textbackslash gamma & 
$\delta$
& \textbackslash delta &
$\epsilon$ & \textbackslash epsilon \\
$\varepsilon$ & \textbackslash varepsilon &
$\zeta$
& \textbackslash zeta &
$\eta$
& \textbackslash eta &
$\theta$
& \textbackslash theta &
$\vartheta$ & \textbackslash vartheta \\
$\iota$
& \textbackslash iota &
$\kappa$
& \textbackslash kappa &
$\lambda$
& \textbackslash lambda &
$\mu$
& \textbackslash mu &
$\nu$
& \textbackslash nu \\
$\xi$
& \textbackslash xi &
$\pi$
& \textbackslash pi &
$\varpi$ & \textbackslash varpi &
$\rho$
& \textbackslash rho &
$\varrho$ & \textbackslash varrho \\
$\sigma$
& \textbackslash sigma &
$\varsigma$ & \textbackslash varsigma &
$\tau$
& \textbackslash tau &
$\upsilon$
& \textbackslash upsilon &
$\phi$
& \textbackslash phi \\
$\varphi$ & \textbackslash varphi &
$\chi$
& \textbackslash chi &
$\psi$
& \textbackslash psi &
$\omega$
& \textbackslash omega &
$\Gamma$
& \textbackslash Gamma \\
$\Delta$
& \textbackslash Delta &
$\Theta$
& \textbackslash Theta &
$\Lambda$
& \textbackslash Lambda &
$\Xi$
& \textbackslash Xi &
$\Pi$
& \textbackslash Pi \\
$\Sigma$
& \textbackslash Sigma &
$\Upsilon$
& \textbackslash Upsilon &
$\Phi$
& \textbackslash Phi &
$\Psi$
& \textbackslash Psi &
$\Omega$
& \textbackslash Omega \\
\end{tabular}

\subsection*{Delimitadores de tamanho variável}
\addcontentsline{toc}{subsection}{Delimitadores de tamanho variável}

Caracteres de agrupamento em diversos tamanhos, além dos tamanhos normais \{|[()]|\}. Estes comandos também só funcionam no modo matemático.

\index{Biggl@\textbackslash Biggl}
\index{biggl@\textbackslash biggl}
\index{Bigl@\textbackslash Bigl}
\index{bigl@\textbackslash bigl}
\index{bigr@\textbackslash bigr}
\index{Bigr@\textbackslash Bigr}
\index{biggr@\textbackslash biggr}
\index{Biggr@\textbackslash Biggr}

\begin{tabular}{r>{\ttfamily\footnotesize}lccr>{\ttfamily\footnotesize}lccr>{\ttfamily\footnotesize}lccr>{\ttfamily\footnotesize}l}
$\Biggl($ & \textbackslash Biggl( & & &
$\biggl($ & \textbackslash biggl( & & &
$\Bigl($ & \textbackslash Bigl( & & &
$\bigl($ & \textbackslash bigl( \\
$\bigr)$ & \textbackslash bigr) & & &
$\Bigr)$ & \textbackslash Bigr) & & &
$\biggr)$ & \textbackslash biggr) & & &
$\Biggr)$ & \textbackslash Biggr) \\
\arrayrulecolor{blue!50}\hdashline
$\Biggl[$ & \textbackslash Biggl[ & & &
$\biggl[$ & \textbackslash biggl[ & & &
$\Bigl[$ & \textbackslash Bigl[ & & &
$\bigl[$ & \textbackslash bigl[ \\
$\bigr]$ & \textbackslash bigr] & & &
$\Bigr]$ & \textbackslash Bigr] & & &
$\biggr]$ & \textbackslash biggr] & & &
$\Biggr]$ & \textbackslash Biggr] \\
\arrayrulecolor{blue!50}\hdashline
$\Biggl|$ & \textbackslash Biggl| & & &
$\biggl|$ & \textbackslash biggl| & & &
$\Bigl|$ & \textbackslash Bigl| & & &
$\bigl|$ & \textbackslash bigl| \\
$\bigr|$ & \textbackslash bigr| & & &
$\Bigr|$ & \textbackslash Bigr| & & &
$\biggr|$ & \textbackslash biggr| & & &
$\Biggr|$ & \textbackslash Biggr| \\
\arrayrulecolor{blue!50}\hdashline
$\Biggl\{$ & \textbackslash Biggl\textbackslash \{ & & &
$\biggl\{$ & \textbackslash biggl\textbackslash \{ & & &
$\Bigl\{$ & \textbackslash Bigl\textbackslash \{ & & &
$\bigl\{$ & \textbackslash bigl\textbackslash \{ \\
$\bigr\}$ & \textbackslash bigr\textbackslash \} & & &
$\Bigr\}$ & \textbackslash Bigr\textbackslash \} & & &
$\biggr\}$ & \textbackslash biggr\textbackslash \} & & &
$\Biggr\}$ & \textbackslash Biggr\textbackslash \} \\
\end{tabular}

\newpage
\section*{Símbolos}
\addcontentsline{toc}{section}{Símbolos}

Existem dezenas de pacotes exclusivos ao fornecimento de símbolos e ícones. Alguns serão aqui apresentados resumidamente.

\subsection*{Nativos do \LaTeX\ \footnotesize{(mas alguns requerem pacote \texttt{latexsym})}}
\addcontentsline{toc}{subsection}{Nativos do \LaTeX}\index{latexsym}

\subsubsection{Operadores \footnotesize{(modo matemático)}}

\noindent
\begin{tabular}{l>{\ttfamily\footnotesize}ll>{\ttfamily\footnotesize}ll>{\ttfamily\footnotesize}ll>{\ttfamily\footnotesize}l}
$\amalg$ & \textbackslash amalg &
$\ast$ & \textbackslash ast &
$\bigcirc$ & \textbackslash bigcirc &
$\bigtriangledown$ & \textbackslash bigtriangledown \\
$\bigtriangleup$ & \textbackslash bigtriangleup &
$\bullet$ & \textbackslash bullet &
$\cdot$ & \textbackslash cdot &
$\circ$ & \textbackslash circ \\
$\dagger$ & \textbackslash dagger &
$\ddagger$ & \textbackslash ddagger &
$\diamond$ &\textbackslash diamond &
$\div$ & \textbackslash div \\
$\lhd$ & \textbackslash lhd &
$\mp$ & \textbackslash mp &
$\odot$ & \textbackslash odot &
$\ominus$ & \textbackslash ominus \\
$\oplus$ & \textbackslash oplus &
$\oslash$ & \textbackslash oslash &
$\otimes$ & \textbackslash otimes &
$\pm$ & \textbackslash pm \\
$\rhd$ & \textbackslash rhd &
$\setminus$ & \textbackslash setminus &
$\sqcap$ & \textbackslash sqcap &
$\sqcup$ & \textbackslash sqcup \\
$\star$ & \textbackslash star &
$\times$ & \textbackslash times &
$\triangleleft$ & \textbackslash triangleleft &
$\triangleright$ & \textbackslash triangleright \\
$\unlhd$ & \textbackslash unlhd &
$\unrhd$ & \textbackslash unrhd &
$\uplus$ & \textbackslash uplus &
$\vee$ & \textbackslash vee \\
$\wedge$ & \textbackslash wedge &
$\wr$ & \textbackslash wr \\
\end{tabular}

\subsubsection{Operadores de tamanho variável \footnotesize{(modo matemático)}}

\noindent
\begin{tabular}{l>{\ttfamily\footnotesize}ll>{\ttfamily\footnotesize}ll>{\ttfamily\footnotesize}ll>{\ttfamily\footnotesize}ll>{\ttfamily\footnotesize}l}
$\bigcap$ & \textbackslash bigcap &
$\bigcup$ & \textbackslash bigcup &
$\bigodot$ & \textbackslash bigodot &
$\bigoplus$ & \textbackslash bigoplus &
$\bigotimes$ & \textbackslash bigotimes \\
$\bigsqcup$ & \textbackslash bigsqcup &
$\biguplus$ & \textbackslash biguplus &
$\bigvee$ & \textbackslash bigvee &
$\bigwedge$ & \textbackslash bigwedge &
$\coprod$ & \textbackslash coprod \\
$\int$ & \textbackslash int &
$\oint$ & \textbackslash oint &
$\prod$ & \textbackslash prod &
$\sum$ & \textbackslash sum \\
\end{tabular}

\subsubsection{Setas \footnotesize{(modo matemático)}}

\noindent
\begin{tabular}{l>{\ttfamily\footnotesize}ll>{\ttfamily\footnotesize}ll>{\ttfamily\footnotesize}l}
$\Downarrow$ & \textbackslash Downarrow &
$\downarrow$ & \textbackslash downarrow &
$\hookleftarrow$ & \textbackslash hookleftarrow \\
$\hookrightarrow$ & \textbackslash hookrightarrow &
$\leadsto$ & \textbackslash leadsto &
$\leftarrow$ & \textbackslash leftarrow \\
$\Leftarrow$ & \textbackslash Leftarrow &
$\Leftrightarrow$ & \textbackslash Leftrightarrow &
$\leftrightarrow$ & \textbackslash leftrightarrow \\
$\longleftarrow$ & \textbackslash longleftarrow &
$\Longleftarrow$ & \textbackslash Longleftarrow &
$\longleftrightarrow$ & \textbackslash longleftrightarrow \\
$\Longleftrightarrow$ & \textbackslash Longleftrightarrow &
$\longmapsto$ & \textbackslash longmapsto &
$\Longrightarrow$ & \textbackslash Longrightarrow \\
$\longrightarrow$ & \textbackslash longrightarrow &
$\mapsto$ & \textbackslash mapsto &
$\nearrow$ & \textbackslash nearrow \\
$\nwarrow$ & \textbackslash nwarrow &
$\Rightarrow$ & \textbackslash Rightarrow &
$\rightarrow$ & \textbackslash rightarrow \\
$\searrow$ & \textbackslash searrow &
$\swarrow$ & \textbackslash swarrow &
$\uparrow$ & \textbackslash uparrow \\
$\Uparrow$ & \textbackslash Uparrow &
$\updownarrow$ & \textbackslash updownarrow &
$\Updownarrow$ & \textbackslash Updownarrow \\
\end{tabular}

\subsubsection{Desigualdades \footnotesize{(modo matemático)}}

\noindent
\begin{tabular}{l>{\ttfamily\footnotesize}lcl>{\ttfamily\footnotesize}lcl>{\ttfamily\footnotesize}lcl>{\ttfamily\footnotesize}lcl>{\ttfamily\footnotesize}l}
$\geq$ & \textbackslash geq & &
$\gg$ & \textbackslash gg & &
$\leq$ & \textbackslash leq & &
$\ll$ & \textbackslash ll & &
$\neq$ & \textbackslash neq \\
\end{tabular}

\subsubsection{Relações binárias \footnotesize{(modo matemático)}}

\noindent
\begin{tabular}{l>{\ttfamily\footnotesize}ll>{\ttfamily\footnotesize}ll>{\ttfamily\footnotesize}ll>{\ttfamily\footnotesize}ll>{\ttfamily\footnotesize}ll>{\ttfamily\footnotesize}l}
$\approx$ & \textbackslash approx &
$\asymp$ & \textbackslash asymp &
$\bowtie$ & \textbackslash bowtie &
$\cong$ & \textbackslash cong &
$\dashv$ & \textbackslash dashv &
$\doteq$ & \textbackslash doteq \\
$\equiv$ & \textbackslash equiv &
$\frown$ & \textbackslash frown &
$\Join$ & \textbackslash Join &
$\mid$ & \textbackslash mid &
$\models$ & \textbackslash models &
$\parallel$ & \textbackslash parallel \\
$\perp$ & \textbackslash perp &
$\prec$ & \textbackslash prec &
$\preceq$ & \textbackslash preceq &
$\propto$ & \textbackslash propto &
$\sim$ & \textbackslash sim &
$\simeq$ & \textbackslash simeq \\
$\smile$ & \textbackslash smile &
$\succ$ & \textbackslash succ &
$\succeq$ & \textbackslash succeq &
$\vdash$ & \textbackslash vdash \\
\end{tabular}

\subsubsection{Funções matemáticas \footnotesize{(modo matemático)}}

\noindent
\begin{tabular}{l>{\ttfamily\footnotesize}ll>{\ttfamily\footnotesize}ll>{\ttfamily\footnotesize}ll>{\ttfamily\footnotesize}ll>{\ttfamily\footnotesize}l}
$\arccos$ & \textbackslash arccos &
$\cos$ & \textbackslash cos &
$\csc$ & \textbackslash csc &
$\exp$ & \textbackslash exp &
$\ker$ & \textbackslash ker \\
$\limsup$ & \textbackslash limsup &
$\min$ & \textbackslash min &
$\sinh$ & \textbackslash sinh &
$\arcsin$ & \textbackslash arcsin &
$\cosh$ & \textbackslash cosh \\
$\deg$ & \textbackslash deg &
$\gcd$ & \textbackslash gcd &
$\lg$ & \textbackslash lg &
$\ln$ & \textbackslash ln &
$\Pr$ & \textbackslash Pr \\
$\sup$ & \textbackslash sup &
$\arctan$ & \textbackslash arctan &
$\cot$ & \textbackslash cot &
$\det$ & \textbackslash det &
$\hom$ & \textbackslash hom \\
$\lim$ & \textbackslash lim &
$\log$ & \textbackslash log &
$\sec$ & \textbackslash sec &
$\tan$ & \textbackslash tan &
$\arg$ & \textbackslash arg \\
$\coth$ & \textbackslash coth &
$\dim$ & \textbackslash dim &
$\inf$ & \textbackslash inf &
$\liminf$ & \textbackslash liminf &
$\max$ & \textbackslash max \\
$\sin$ & \textbackslash sin &
$\tanh$ & \textbackslash tanh \\
\end{tabular}

\subsubsection{Acentos extensíveis \footnotesize{(modo matemático)}}

\noindent
\begin{tabular}{l>{\ttfamily\footnotesize}ll>{\ttfamily\footnotesize}ll>{\ttfamily\footnotesize}l}
$\widetilde{abc}$ & \textbackslash widetilde\{abc\} &
$\widehat{abc}$ & \textbackslash widehat\{abc\} &
$\overleftarrow{abc}$ & \textbackslash overleftarrow\{abc\} \\
$\overrightarrow{abc}$ & \textbackslash overrightarrow\{abc\} &
$\overline{abc}$ & \textbackslash overline\{abc\} &
$\underline{abc}$ & \textbackslash underline\{abc\} \\
$\overbrace{abc}$ & \textbackslash overbrace\{abc\} &
$\underbrace{abc}$ & \textbackslash underbrace\{abc\} &
$\sqrt{abc}$ & \textbackslash sqrt\{abc\} \\
\end{tabular}

\subsubsection{Relações de conjuntos \footnotesize{(modo matemático)}}

\noindent
\begin{tabular}{l>{\ttfamily\footnotesize}lcl>{\ttfamily\footnotesize}lcl>{\ttfamily\footnotesize}lcl>{\ttfamily\footnotesize}l}
$\in$ & \textbackslash in & &
$\ni$ & \textbackslash ni & &
$\cap$ & \textbackslash cap & &
$\cup$ & \textbackslash cup \\
$\subset$ & \textbackslash subset & &
$\supset$ & \textbackslash supset & &
$\subseteq$ & \textbackslash subseteq & &
$\supseteq$ & \textbackslash supseteq \\
$\sqsubset$ & \textbackslash sqsubset & &
$\sqsupset$ & \textbackslash sqsupset & &
$\sqsubseteq$ & \textbackslash sqsubseteq & &
$\sqsupseteq$ & \textbackslash sqsupseteq \\
$\exists$ & \textbackslash exists & &
$\forall$ & \textbackslash forall \\
\end{tabular}

\subsubsection{Símbolos diversos \footnotesize{(modo matemático)}}

\noindent
\begin{tabular}{l>{\ttfamily\footnotesize}lcl>{\ttfamily\footnotesize}lcl>{\ttfamily\footnotesize}lcl>{\ttfamily\footnotesize}l}
$\bot$ & \textbackslash bot & &
$\ell$ & \textbackslash ell & &
$\hbar$ & \textbackslash hbar & &
$\Im$ & \textbackslash Im \\
$\imath$ & \textbackslash imath & &
$\jmath$ & \textbackslash jmath & &
$\partial$ & \textbackslash partial & &
$\Re$ & \textbackslash Re \\
$\top$ & \textbackslash top & &
$\wp$ & \textbackslash wp & &
$\aleph$ & \textbackslash aleph & &
$\emptyset$ & \textbackslash emptyset \\
$\angle$ & \textbackslash angle & &
$\backslash$ & \textbackslash backslash & &
$\Box$ & \textbackslash Box & &
$\Diamond$ & \textbackslash Diamond \\
$\infty$ & \textbackslash infty & &
$\mho$ & \textbackslash mho & &
$\nabla$ & \textbackslash nabla & &
$\neg$ & \textbackslash neg \\
$\prime$ & \textbackslash prime & &
$\surd$ & \textbackslash surd & &
$\triangle$ & \textbackslash triangle & &
$\flat$ & \textbackslash flat \\
$\natural$ & \textbackslash natural & &
$\sharp$ & \textbackslash sharp & &
$\clubsuit$ & \textbackslash clubsuit & &
$\diamondsuit$ & \textbackslash diamondsuit \\
$\heartsuit$ & \textbackslash heartsuit & &
$\spadesuit$ & \textbackslash spadesuit \\
\end{tabular}

\subsubsection{Modo texto}

\noindent
\begin{tabular}{l>{\ttfamily\footnotesize}ll>{\ttfamily\footnotesize}ll>{\ttfamily\footnotesize}l}
\textbackslash & \textbackslash textbackslash &
\textbar & \textbackslash textbar &
\textbardbl & \textbackslash textbardbl \\
\textbigcircle & \textbackslash textbigcircle &
\textbullet & \textbackslash textbullet &
\textdagger & \textbackslash textdagger \\
\textdaggerdbl & \textbackslash textdaggerdbl &
\textellipsis & \textbackslash textellipsis &
\textemdash & \textbackslash textemdash \\
\textendash & \textbackslash textendash &
\textexclamdown & \textbackslash textexclamdown &
\textgreater & \textbackslash textgreater \\
\textless & \textbackslash textless &
\textordfeminine & \textbackslash textordfeminine &
\textordmasculine & \textbackslash \textordmasculine \\
\textparagraph & \textbackslash textparagraph &
\textperiodcentered & \textbackslash textperiodcentered &
\textpertenthousand & \textbackslash textpertenthousand \\
\textperthousand & \textbackslash textperthousand &
\textquestiondown & \textbackslash textquestiondown &
\textquotedblleft & \textbackslash textquotedblleft \\
\textquotedblright & \textbackslash textquotedblright &
\textquoteleft & \textbackslash textquoteleft &
\textquoteright & \textbackslash textquoteright \\
\textsection & \textbackslash textsection &
\textunderscore & \textbackslash textunderscore &
\textvisiblespace & \textbackslash textvisiblespace \\
\end{tabular}

\subsection*{Text Companion \footnotesize{(pacote \texttt{textcomp})}}
\addcontentsline{toc}{subsection}{Text Companion}\index{textcomp}

Obs.: Estes comandos funcionam no modo texto.

\noindent
\begin{tabular}{l>{\ttfamily\footnotesize}ll>{\ttfamily\footnotesize}ll>{\ttfamily\footnotesize}l}
\textdownarrow & \textbackslash textdownarrow &
\textrightarrow & \textbackslash textrightarrow &
\textleftarrow & \textbackslash textleftarrow \\
\textuparrow & \textbackslash textuparrow &
\textdollar & \textbackslash textdollar &
\textsterling & \textbackslash textsterling \\
\texteuro & \textbackslash texteuro &
\textcent & \textbackslash textcent &
\textwon & \textbackslash textwon \\
\textyen & \textbackslash textyen &
\textpeso & \textbackslash textpeso &
\textcurrency & \textbackslash textcurrency \\
\textcircledP & \textbackslash textcircledP &
\textcopyright & \textbackslash textcopyright &
\textregistered & \textbackslash textregistered \\
\textservicemark & \textbackslash textservicemark &
\texttrademark & \textbackslash texttrademark &
\textcelsius & \textbackslash textcelsius \\
\textmho & \textbackslash textmho &
\textmu & \textbackslash textmu &
\textohm & \textbackslash textohm \\
\textacutedbl & \textbackslash textacutedbl &
\textasciicaron & \textbackslash textasciicaron &
\textasciimacron & \textbackslash textasciimacron \\
\textasciiacute & \textbackslash textasciiacute &
\textasciidieresis & \textbackslash textasciidieresis &
\textgravedbl & \textbackslash textgravedbl \\
\textasciibreve & \textbackslash textasciibreve &
\textasciigrave & \textbackslash textasciigrave &
\textbrokenbar & \textbackslash textbrokenbar \\
\textdiscount & \textbackslash textdiscount &
\textestimated & \textbackslash textestimated &
\textnumero & \textbackslash textnumero \\
\textopenbullet & \textbackslash textopenbullet &
\textquotesingle & \textbackslash textquotesingle &
\textquotestraightbase & \textbackslash textquotestraightbase \\
\textquotestraightdblbase & \textbackslash textquotestraightdblbase &
\textrecipe & \textbackslash textrecipe &
\textreferencemark & \textbackslash textreferencemark \\
\texttildelow & \textbackslash texttildelow &
\textblank & \textbackslash textblank &
\textpilcrow & \textbackslash textpilcrow \\
\textlangle & \textbackslash textlangle &
\textrangle & \textbackslash textrangle &
\textlbrackdbl & \textbackslash textlbrackdbl \\
\textrbrackdbl & \textbackslash textrbrackdbl &
\textlquill & \textbackslash textlquill &
\textrquill & \textbackslash textrquill \\
\textdegree & \textbackslash textdegree &
\textlnot & \textbackslash textlnot &
\textminus & \textbackslash textminus \\
\texttimes & \textbackslash texttimes &
\textdiv & \textbackslash textdiv &
\textpm & \textbackslash textpm \\
\textonesuperior & \textbackslash textonesuperior &
\texttwosuperior & \textbackslash texttwosuperior &
\textthreesuperior & \textbackslash textthreesuperior \\
\textsurd & \textbackslash textsurd &
\textmusicalnote & \textbackslash textmusicalnote &
\textborn & \textbackslash textborn \\
\textdied & \textbackslash textdied &
\textmarried & \textbackslash textmarried &
\textdivorced & \textbackslash textdivorced \\
\end{tabular}

\newpage
\subsection*{American Mathematical Society \footnotesize{(pacotes \texttt{amsmath} e \texttt{amssymb})}}
\addcontentsline{toc}{subsection}{American Mathematical Society}\index{amsmath}\index{amssymb}

Obs.: Estes comandos funcionam no modo matemático.

\noindent
\begin{tabular}{l>{\ttfamily\footnotesize}ll>{\ttfamily\footnotesize}ll>{\ttfamily\footnotesize}l}
$\leftarrowtail$ & \textbackslash leftarrowtail &
$\rightarrowtail$ & \textbackslash rightarrowtail &
$\looparrowleft$ & \textbackslash looparrowleft \\
$\looparrowright$ & \textbackslash looparrowright &
$\Rsh$ & \textbackslash Rsh &
$\Lsh$ & \textbackslash Lsh \\
$\curvearrowleft$ & \textbackslash curvearrowleft &
$\curvearrowright$ & \textbackslash curvearrowright &
$\circlearrowleft$ & \textbackslash circlearrowleft \\
$\circlearrowright$ & \textbackslash circlearrowright &
$\upharpoonright$ & \textbackslash upharpoonright &
$\upharpoonleft$ & \textbackslash upharpoonleft \\
$\downharpoonright$ & \textbackslash downharpoonright &
$\downharpoonleft$ & \textbackslash downharpoonleft &
$\rightleftarrows$ & \textbackslash rightleftarrows \\
$\leftrightarrows$ & \textbackslash leftrightarrows &
$\leftleftarrows$ & \textbackslash leftleftarrows &
$\upuparrows$ & \textbackslash upuparrows \\
$\rightrightarrows$ & \textbackslash rightrightarrows &
$\downdownarrows$ & \textbackslash downdownarrows &
$\leftrightharpoons$ & \textbackslash leftrightharpoons \\
$\rightleftharpoons$ & \textbackslash rightleftharpoons &
$\Lleftarrow$ & \textbackslash Lleftarrow &
$\Rrightarrow$ & \textbackslash Rrightarrow \\
$\nexists$ & \textbackslash nexists &
$\varnothing$ & \textbackslash varnothing &
$\measuredangle$ & \textbackslash measuredangle \\
$\sphericalangle$ & \textbackslash sphericalangle &
$\nmid$ & \textbackslash nmid &
$\nparallel$ & \textbackslash nparallel \\
$\wedge$ & \textbackslash wedge &
$\therefore$ & \textbackslash therefore &
$\because$ & \textbackslash because \\
$\backsim$ & \textbackslash backsim &
$\wr$ & \textbackslash wr &
$\nsim$ & \textbackslash nsim \\
$\eqsim$ & \textbackslash eqsim &
$\ncong$ & \textbackslash ncong &
$\approxeq$ & \textbackslash approxeq \\
$\leqq$ & \textbackslash leqq &
$\geqq$ & \textbackslash geqq &
$\lneqq$ & \textbackslash lneqq \\
$\gneqq$ & \textbackslash gneqq &
$\between$ & \textbackslash between &
$\nless$ & \textbackslash nless \\
$\ngtr$ & \textbackslash ngtr &
$\nleq$ & \textbackslash nleq &
$\ngeq$ & \textbackslash ngeq \\
$\lesssim$ & \textbackslash lesssim &
$\gtrsim$ & \textbackslash gtrsim &
$\lessgtr$ & \textbackslash lessgtr \\
$\gtrless$ & \textbackslash gtrless &
$\nsubseteq$ & \textbackslash nsubseteq &
$\nsupseteq$ & \textbackslash nsupseteq \\
$\subsetneq$ & \textbackslash subsetneq &
$\supsetneq$ & \textbackslash supsetneq &
$\circledcirc$ & \textbackslash circledcirc \\
$\boxplus$ & \textbackslash boxplus &
$\boxminus$ & \textbackslash boxminus &
$\boxtimes$ & \textbackslash boxtimes \\
$\boxdot$ & \textbackslash boxdot &
$\dashv$ & \textbackslash dashv &
$\vDash$ & \textbackslash vDash \\
$\Vdash$ & \textbackslash Vdash &
$\Vvdash$ & \textbackslash Vvdash &
$\nvdash$ & \textbackslash nvdash \\
$\nvDash$ & \textbackslash nvDash &
$\nVdash$ & \textbackslash nVdash &
$\nVDash$ & \textbackslash nVDash \\
$\vartriangleleft$ & \textbackslash vartriangleleft &
$\vartriangleright$ & \textbackslash vartriangleright &
$\multimap$ & \textbackslash multimap \\
$\intercal$ & \textbackslash intercal &
$\ltimes$ & \textbackslash ltimes &
$\rtimes$ & \textbackslash rtimes \\
$\leftthreetimes$ & \textbackslash leftthreetimes &
$\rightthreetimes$ & \textbackslash rightthreetimes &
$\backsimeq$ & \textbackslash backsimeq \\
$\curlyvee$ & \textbackslash curlyvee &
$\curlywedge$ & \textbackslash curlywedge &
$\lessdot$ & \textbackslash lessdot \\
$\gtrdot$ & \textbackslash gtrdot &
$\lll$ & \textbackslash lll &
$\ggg$ & \textbackslash ggg \\
$\lesseqgtr$ & \textbackslash lesseqgtr &
$\gtreqless$ & \textbackslash gtreqless &
%$\Diamond$ & \textbackslash Diamond \\
%$\lozenge$ & \textbackslash lozenge &
%$\square$ & \textbackslash square &
%$\blacksquare$ & \textbackslash blacksquare \\
%$\bigstar$ & \textbackslash bigstar &
$\Join$ & \textbackslash Join \\
$\leqslant$ & \textbackslash leqslant &
$\geqslant$ & \textbackslash geqslant &
$\lessapprox$ & \textbackslash lessapprox \\
$\gtrapprox$ & \textbackslash gtrapprox &
$\lneq$ & \textbackslash lneq &
$\gneq$ & \textbackslash gneq \\
$\lnapprox$ & \textbackslash lnapprox &
$\gnapprox$ & \textbackslash gnapprox &
$\lesseqqgtr$ & \textbackslash lesseqqgtr \\
$\gtreqqless$ & \textbackslash gtreqqless &
$\subseteqq$ & \textbackslash subseteqq &
$\supseteqq$ & \textbackslash supseteqq \\
$\subsetneqq$ & \textbackslash subsetneqq &
$\supsetneqq$ & \textbackslash supsetneqq &
$\varkappa$ & \textbackslash varkappa \\
$\implies$ & \textbackslash implies &
$\checkmark$ & \textbackslash checkmark &
%$\circledR$ & \textbackslash circledR &
$\maltese$ & \textbackslash maltese \\
$\ulcorner$ & \textbackslash ulcorner &
$\urcorner$ & \textbackslash urcorner &
$\llcorner$ & \textbackslash llcorner \\
$\lrcorner$ & \textbackslash lrcorner &
$\mathbb{N}$ & \textbackslash mathbb\{N\} &
$\mathbb{Z}$ & \textbackslash mathbb\{Z\} \\
$\mathbb{Q}$ & \textbackslash mathbb\{Q\} &
$\mathbb{R}$ & \textbackslash mathbb\{R\} &
$\mathbb{C}$ & \textbackslash mathbb\{C\} \\
\end{tabular}

Dica: O pacote \texttt{\{amsmath\}} tem comandos para formatação do texto dentro do modo matemático. O comando \texttt{\textbackslash text\{\}} aplica a formatação do ambiente externo ao modo matemático. E os comandos \texttt{\textbackslash mathrm\{\}}, \texttt{\textbackslash mathsf\{\}}, \texttt{\textbackslash mathbf\{\}}, \texttt{\textbackslash mathtt\{\}} e \texttt{\textbackslash mathit\{\}} especificam o estilo e família.

\subsection*{Font Awesome \footnotesize{(pacote \texttt{fontawesome})}}
\addcontentsline{toc}{subsection}{Font Awesome}\index{fontawesome}

\noindent
\begin{tabular}{l>{\ttfamily\footnotesize}ll>{\ttfamily\footnotesize}ll>{\ttfamily\footnotesize}l}
\faFacebook & \textbackslash faFacebook &
\faInstagram & \textbackslash faInstagram &
\faTwitter & \textbackslash faTwitter \\
\faLinkedin & \textbackslash faLinkedin &
\faPinterest & \textbackslash faPinterest &
\faReddit & \textbackslash faReddit \\
\faFoursquare & \textbackslash faFoursquare &
\faTumblr & \textbackslash faTumblr &
\faAmazon & \textbackslash faAmazon \\
\faVimeo & \textbackslash faVimeo &
\faYoutube & \textbackslash faYoutube &
\faYelp & \textbackslash faYelp \\
\faDropbox & \textbackslash faDropbox &
\faGithub & \textbackslash faGithub &
\faGoogle & \textbackslash faGoogle \\
\faSkype & \textbackslash faSkype &
\faLinux & \textbackslash faLinux &
\faAndroid & \textbackslash faAndroid \\
\faWindows & \textbackslash faWindows &
\faApple & \textbackslash faApple &
\faFirefox & \textbackslash faFirefox \\
\faChrome & \textbackslash faChrome &
\faOpera & \textbackslash faOpera &
\faInternetExplorer & \textbackslash faInternetExplorer \\
\faFloppyO & \textbackslash faFloppyO &
\faHddO & \textbackslash faHddO &
\faMousePointer & \textbackslash faMousePointer \\
\faSpotify & \textbackslash faSpotify &
\faSoundcloud & \textbackslash faSoundcloud &
\faHeadphones & \textbackslash faHeadphones \\
\faMicrophone & \textbackslash faMicrophone &
\faThumbsOUp & \textbackslash faThumbsOUp &
\faThumbsODown & \textbackslash faThumbsODown \\
\faHandORight & \textbackslash faHandORight &
\faHandOLeft & \textbackslash faHandOLeft &
\faArrowDown & \textbackslash faArrowDown \\
\faArrowUp & \textbackslash faArrowUp &
\faArrowLeft & \textbackslash faArrowLeft &
\faArrowRight & \textbackslash faArrowRight \\
\faChevronCircleDown & \textbackslash faChevronCircleDown &
\faChevronCircleUp & \textbackslash faChevronCircleUp &
\faChevronCircleLeft & \textbackslash faChevronCircleLeft \\
\faChevronCircleRight & \textbackslash faChevronCircleRight &
\faStar & \textbackslash faStar &
\faStarHalf & \textbackslash faStarHalf \\
\faStarHalfO & \textbackslash faStarHalfO &
\faStarO & \textbackslash faStarO &
\faCircle & \textbackslash faCircle \\
\faCircleO & \textbackslash faCircleO &
\faSquare & \textbackslash faSquare &
\faSquareO & \textbackslash faSquareO \\
\faMale & \textbackslash faMale &
\faFemale & \textbackslash faFemale &
\faCheck & \textbackslash faCheck \\
\faClose & \textbackslash faClose &
\faRecycle & \textbackslash faRecycle &
\faPowerOff & \textbackslash faPowerOff \\
\faSignal & \textbackslash faSignal &
\faWifi & \textbackslash faWifi &
\faBatteryEmpty & \textbackslash faBatteryEmpty \\
\faBatteryFull & \textbackslash faBatteryFull &
\faBatteryHalf & \textbackslash faBatteryHalf &
\faBatteryQuarter & \textbackslash faBatteryQuarter \\
\faSortAlphaAsc & \textbackslash faSortAlphaAsc &
\faSortAlphaDesc & \textbackslash faSortAlphaDesc &
\faSortNumericAsc & \textbackslash faSortNumericAsc \\
\faSortNumericDesc & \textbackslash faSortNumericDesc &
\faCcVisa & \textbackslash faCcVisa &
\faCcMastercard & \textbackslash faCcMastercard \\
\faCcAmex & \textbackslash faCcAmex &
\faCcDinersClub & \textbackslash faCcDinersClub &
\faFutbolO & \textbackslash faFutbolO \\
\faScissors & \textbackslash faScissors &
\faPhone & \textbackslash faPhone &
\faShoppingCart & \textbackslash faShoppingCart \\
\faAngleDown & \textbackslash faAngleDown &
\faAngleLeft & \textbackslash faAngleLeft &
\faAngleRight & \textbackslash faAngleRight \\
\faAngleUp & \textbackslash faAngleUp &
\faAngleDoubleDown & \textbackslash faAngleDoubleDown &
\faAngleDoubleLeft & \textbackslash faAngleDoubleLeft \\
\faAngleDoubleRight & \textbackslash faAngleDoubleRight &
\faAngleDoubleUp & \textbackslash faAngleDoubleUp &
\faRefresh & \textbackslash faRefresh \\
\faBan & \textbackslash faBan &
\faRocket & \textbackslash faRocket &
\faRss & \textbackslash faRss \\
\faSearch & \textbackslash faSearch &
\faSearchMinus & \textbackslash faSearchMinus &
\faSearchPlus & \textbackslash faSearchPlus \\
\faShare & \textbackslash faShare &
\faShareAlt & \textbackslash faShareAlt &
\faBell & \textbackslash faBell \\
\faBellO & \textbackslash faBellO &
\faShield & \textbackslash faShield &
\faBinoculars & \textbackslash faBinoculars \\
\faBolt & \textbackslash faBolt &
\faBook & \textbackslash faBook &
\faSliders & \textbackslash faSliders \\
\faHeart & \textbackslash faHeart &
\faHeartbeat & \textbackslash faHeartbeat &
\faHeartO & \textbackslash faHeartO \\
\faHistory & \textbackslash faHistory &
\faHome & \textbackslash faHome &
\faCalendar & \textbackslash faCalendar \\
\faCalculator & \textbackslash faCalculator &
\faHourglassEnd & \textbackslash faHourglassEnd &
\faHourglassHalf & \textbackslash faHourglassHalf \\
\faHourglassStart & \textbackslash faHourglassStart &
\faCamera & \textbackslash faCamera &
\faCar & \textbackslash faCar \\
\faBicycle & \textbackslash faBicycle &
\faMotorcycle & \textbackslash faMotorcycle &
\faBus & \textbackslash faBus \\
\faTrain & \textbackslash faTrain &
\faAmbulance & \textbackslash faAmbulance &
\faPlane & \textbackslash faPlane \\
\faTree & \textbackslash faTree &
\faUmbrella & \textbackslash faUmbrella &
\faDiamond & \textbackslash faDiamond \\
\faDatabase & \textbackslash faDatabase &
\faCube & \textbackslash faCube &
\faDownload & \textbackslash faDownload \\
\faEnvelope & \textbackslash faEnvelope &
\faEnvelopeO & \textbackslash faEnvelopeO &
\faPaperclip & \textbackslash faPaperclip \\
\faPaperPlane & \textbackslash faPaperPlane &
\faPaperPlaneO & \textbackslash faPaperPlaneO &
\faPaw & \textbackslash faPaw \\
\faPlay & \textbackslash faPlay &
\faStop & \textbackslash faStop &
\faBackward & \textbackslash faBackward \\
\faForward & \textbackslash faForward &
\faFastBackward & \textbackslash faFastBackward &
\faFastForward & \textbackslash faFastForward \\
\end{tabular}

\subsection*{Zapf Dingbats \footnotesize{(pacote \texttt{pifont})}}
\addcontentsline{toc}{subsection}{Zapf Dingbats}\index{pifont}

Obs.: Estes comandos funcionam no modo texto.

\noindent
\begin{tabular}{l>{\ttfamily\footnotesize}ll>{\ttfamily\footnotesize}ll>{\ttfamily\footnotesize}ll>{\ttfamily\footnotesize}ll>{\ttfamily\footnotesize}l}
\ding{34} & \textbackslash ding\{34\} &
\ding{36} & \textbackslash ding\{36\} &
\ding{42} & \textbackslash ding\{42\} &
\ding{43} & \textbackslash ding\{43\} &
\ding{44} & \textbackslash ding\{44\} \\
\ding{45} & \textbackslash ding\{45\} &
\ding{46} & \textbackslash ding\{46\} &
\ding{47} & \textbackslash ding\{47\} &
\ding{48} & \textbackslash ding\{48\} &
\ding{51} & \textbackslash ding\{51\} \\
\ding{52} & \textbackslash ding\{52\} &
\ding{53} & \textbackslash ding\{53\} &
\ding{54} & \textbackslash ding\{54\} &
\ding{55} & \textbackslash ding\{55\} &
\ding{56} & \textbackslash ding\{56\} \\
\ding{58} & \textbackslash ding\{58\} &
\ding{61} & \textbackslash ding\{61\} &
\ding{62} & \textbackslash ding\{62\} &
\ding{63} & \textbackslash ding\{63\} &
\ding{64} & \textbackslash ding\{64\} \\
\ding{65} & \textbackslash ding\{65\} &
\ding{70} & \textbackslash ding\{70\} &
\ding{71} & \textbackslash ding\{71\} &
\ding{72} & \textbackslash ding\{72\} &
\ding{73} & \textbackslash ding\{73\} \\
\ding{86} & \textbackslash ding\{86\} &
\ding{87} & \textbackslash ding\{87\} &
\ding{88} & \textbackslash ding\{88\} &
\ding{89} & \textbackslash ding\{89\} &
\ding{108} & \textbackslash ding\{108\} \\
\ding{109} & \textbackslash ding\{109\} &
\ding{110} & \textbackslash ding\{110\} &
\ding{111} & \textbackslash ding\{111\} &
\ding{112} & \textbackslash ding\{112\} &
\ding{113} & \textbackslash ding\{113\} \\
\ding{114} & \textbackslash ding\{114\} &
\ding{123} & \textbackslash ding\{123\} &
\ding{124} & \textbackslash ding\{124\} &
\ding{125} & \textbackslash ding\{125\} &
\ding{126} & \textbackslash ding\{126\} \\
\ding{168} & \textbackslash ding\{168\} &
\ding{169} & \textbackslash ding\{169\} &
\ding{170} & \textbackslash ding\{170\} &
\ding{171} & \textbackslash ding\{171\} &
\ding{192} & \textbackslash ding\{192\} \\
\ding{193} & \textbackslash ding\{193\} &
\ding{194} & \textbackslash ding\{194\} &
\ding{195} & \textbackslash ding\{195\} &
\ding{196} & \textbackslash ding\{196\} &
\ding{197} & \textbackslash ding\{197\} \\
\ding{198} & \textbackslash ding\{198\} &
\ding{199} & \textbackslash ding\{199\} &
\ding{200} & \textbackslash ding\{200\} &
\ding{201} & \textbackslash ding\{201\} &
\ding{202} & \textbackslash ding\{202\} \\
\ding{203} & \textbackslash ding\{203\} &
\ding{204} & \textbackslash ding\{204\} &
\ding{205} & \textbackslash ding\{205\} &
\ding{206} & \textbackslash ding\{206\} &
\ding{207} & \textbackslash ding\{207\} \\
\ding{208} & \textbackslash ding\{208\} &
\ding{209} & \textbackslash ding\{209\} &
\ding{210} & \textbackslash ding\{210\} &
\ding{211} & \textbackslash ding\{211\} &
\ding{212} & \textbackslash ding\{212\} \\
\ding{213} & \textbackslash ding\{213\} &
\ding{214} & \textbackslash ding\{214\} &
\ding{215} & \textbackslash ding\{215\} &
\ding{220} & \textbackslash ding\{220\} &
\ding{221} & \textbackslash ding\{221\} \\
\ding{222} & \textbackslash ding\{222\} &
\ding{223} & \textbackslash ding\{223\} &
\ding{224} & \textbackslash ding\{224\} &
\ding{226} & \textbackslash ding\{226\} &
\ding{227} & \textbackslash ding\{227\} \\
\ding{228} & \textbackslash ding\{228\} &
\ding{229} & \textbackslash ding\{229\} &
\ding{233} & \textbackslash ding\{233\} &
\ding{234} & \textbackslash ding\{234\} &
\ding{235} & \textbackslash ding\{235\} \\
\ding{236} & \textbackslash ding\{236\} &
\ding{237} & \textbackslash ding\{237\} &
\ding{238} & \textbackslash ding\{238\} &
\ding{239} & \textbackslash ding\{239\} &
\ding{241} & \textbackslash ding\{241\} \\
\ding{247} & \textbackslash ding\{247\} &
\ding{248} & \textbackslash ding\{248\} &
\ding{249} & \textbackslash ding\{249\} &
\ding{252} & \textbackslash ding\{252\} &
\ding{254} & \textbackslash ding\{254\} \\
\end{tabular}

Dica: Este pacote \texttt{\{pifont\}} traz comandos interessantes. Dois comandos de ambiente \texttt{\{dinglist\}}\index{dinglist} e \texttt{\{dingautolist\}}\index{dingautolist} que constrõem listas rotuladas com os próprios caracteres da fonte. E dois comandos para preenchimento linear \texttt{\textbackslash dingfill}\index{dingfill@\textbackslash dingfill} e \texttt{\textbackslash dingline}\index{dingline@\textbackslash dingline}. Veja exemplos:

\begin{multicols}{2}
\begin{small}
\begin{verbatim}
   \begin{dinglist}{43}
      \item Livros
      \item Revistas
      \item Jornais
   \end{dinglist}
\end{verbatim}
\end{small}

\begin{dinglist}{43}
	\item Livros
	\item Revistas
	\item Jornais
\end{dinglist}
\columnbreak
\begin{small}
\begin{verbatim}
   \begin{dingautolist}{192}
      \item Livros
      \item Revistas
      \item Jornais
   \end{dingautolist}
\end{verbatim}
\end{small}

\begin{dingautolist}{192}
	\item Livros
	\item Revistas
	\item Jornais
\end{dingautolist}
\end{multicols}

O \texttt{\textbackslash dingfill\{n\}} preenche a linha \dingfill{226} com o símbolo.

O \texttt{\textbackslash dingline\{n\}} cria uma nova linha com o símbolo escolhido:

\dingline{34}

\chapter*{Conclusão}
\addcontentsline{toc}{chapter}{Conclusão}

\section*{Considerações}
\addcontentsline{toc}{section}{Considerações}

O \LaTeX\ possui zilhões de comandos e quase sempre há mais de uma solução possível para formatar um conteúdo. Entretanto, o \LaTeX padrão não possui todos os comandos instalados, trazendo apenas o básico, sendo assim, existe a necessidade desta adição de pacotes. Os pacotes implementam novos comandos ao \LaTeX.

Coloquei alguns pacotes neste arquivo e nem todos estão sendo usados nos comandos. Estão aqui somente para você saber que eles existem, pois são relativamente famosos. Porém, muitos exemplos de tipografia e diagramação não estão aqui, o \LaTeX\ é capaz de muito mais!

Quanto aos pacotes, é muito comum ter pacotes que são aprimoramentos ou reescrita de outros. Alguns pacotes pedem até uma certa ordem no carregamento. O manual de cada um deles esclarece estes detalhes. Consulte o manual de cada pacote, para conhecer mais possibilidades de edição em \LaTeX.

\section*{Onde saber mais} \label{sec:outrossites}
\addcontentsline{toc}{section}{Onde saber mais}

%\begin{itemize}
\begin{dingautolist}{192}
\setlength\itemsep{0em}
\item \href{http://www.tex.uniyar.ac.ru/doc/latex2e.pdf}{\LaTeXe\ The macro package for \TeX}
\item \href{https://www.ctan.org}{Comprehensive TeX Archive Network (CTAN)}
\item \href{http://www.tug.org/}{TeX Users Group (TUG)}
\item \href{https://tobi.oetiker.ch/lshort/lshort.pdf}{The Not So Short Introduction to \LaTeXe}
\item \href{http://latex.silmaril.ie/formattinginformation/index.html}{Formatting Information - An introduction to typesetting with \LaTeX}
\item \href{https://latexref.xyz/dev/latex2e.pdf}{\LaTeXe: An unofficial reference manual}
\item \href{https://www.maths.tcd.ie/~dwilkins/LaTeXPrimer/GSWLaTeX.pdf}{Getting Started with \LaTeX}
\item\href{https://dickimaw-books.com/latex/novices/novices-report.pdf}{\LaTeX\ for Complete Novices}
\item \href{http://linorg.usp.br/CTAN/info/lshort/portuguese/pt-lshort.pdf}{Uma não tão pequena introdução ao \LaTeXe}
\item \href{http://linorg.usp.br/CTAN/info/symbols/comprehensive/symbols-a4.pdf}{The Comprehensive \LaTeX\ Symbol List}
\end{dingautolist}
%\end{itemize}

\begin{comment}

\begin{center}
\scalebox{0.60} {
\begin{tikzpicture}
\genealogytree[template=signpost] {
	parent{
		g[neuter]{\huge{filha}}
		c[neuter]{\huge{filha}}
		c[neuter]{\huge{filho}}
		parent{
			g[neuter]{\huge{pai}}
			p[neuter]{\huge{avô}}
			p[neuter]{\huge{avó}}
		}
		p[neuter]{\huge{mãe}}
	}
}
\end{tikzpicture}
}
\end{center}


\tikzstyle{retangulo} = [rectangle, minimum width=3cm, minimum height=2.2cm, text centered, text width=3cm, draw=black, fill=lightgray!30, rounded corners]
\tikzstyle{traco} = [thick,-]

\begin{center}
\scalebox{0.50} {
\begin{tikzpicture} [node distance=2.5cm]
\node (membro1) [retangulo] {
	\textcolor{teal!70!black}{\large{\textbf{Avó A}}} \par \textcolor{purple!70!black}{\large{\textbf{Esposa D}}} \par \textcolor{olive!70!black}{\large{\textbf{Mãe E}}}};
\node (membro2) [retangulo, below of=membro1] {
	\textcolor{purple!70!black}{\large{\textbf{Marido C}}} \par \textcolor{blue!70!black}{\large{\textbf{Pai I}}} \par \textcolor{olive!70!black}{\large{\textbf{Filho E}}} \par \textcolor{violet!70!black}{\large{\textbf{Irmão G}}}};
\node (membro3) [retangulo, right of=membro2, xshift=4cm] {
	\textcolor{teal!70!black}{\large{\textbf{Avó B}}} \par \textcolor{purple!70!black}{\large{\textbf{Esposa C}}} \par \textcolor{olive!70!black}{\large{\textbf{Mãe F}}}};
\node (membro4) [retangulo, below of=membro2, xshift=3.5cm] {
	\textcolor{teal!70!black}{\large{\textbf{Neta A}}} \par \textcolor{blue!70!black}{\large{\textbf{Filha I}}} \par
	\textcolor{black!80}{\large{\textbf{Solteira}}} \par
	\textcolor{violet!70!black}{\large{\textbf{Irmã H}}}};
\node (membro5) [retangulo, below of=membro3, yshift=-2.5cm] {
	\textcolor{purple!70!black}{\large{\textbf{Marido D}}} \par \textcolor{blue!70!black}{\large{\textbf{Pai J}}} \par \textcolor{olive!70!black}{\large{\textbf{Filho F}}} \par \textcolor{violet!70!black}{\large{\textbf{Irmão H}}}};
\node (membro6) [retangulo, below of=membro5, xshift=-5cm] {
	\textcolor{teal!70!black}{\large{\textbf{Neta B}}} \par \textcolor{blue!70!black}{\large{\textbf{Filha J}}} \par
	\textcolor{black!80}{\large{\textbf{Solteira}}} \par
	\textcolor{violet!70!black}{\large{\textbf{Irmã G}}}};

\draw [traco] (membro1) -- (membro2);
\draw [traco] (membro2) -- (membro3);
\draw [traco] (membro2) -| (membro4);
\draw [traco] (membro3) -- (membro5);
\draw [traco] (membro1) -- (-2,0) |- (membro5);
\draw [traco] (membro5) -| (membro6);
\end{tikzpicture}
}
\end{center}

\end{comment}

\chapter*{Colofão}
\addcontentsline{toc}{chapter}{Colofão}

Este eBook foi desenvolvido usando o sistema de preparação de documento \LaTeXe\ e editado com o software TeXstudio no sistema operacional Linux Fedora. O documento foi convertido para o formato PDF pelo TeX Live com o pdfTeX. O corpo do texto utiliza a fonte Computer Modern, no tamanho 12pt e as páginas possuem o tamanho A4, com três centímetros nas margens superior e esquerda, e com 2,5 centímetros nas margens direita e inferior.

As imagens da capa e da contracapa foram obtidas pelo website Pexels.com. A fotografia da capa foi obtida no endereço https:// www.pexels.com/ photo/ photo-of-clouds-during-daytime-2088205/ e a fotografia da contracapa foi obtida no endereço https:// www.pexels.com/ photo/ photo-of-airplane-with-smoke-trail-2088203/. Ambas estão creditadas a Eberhard Grossgasteiger.

O código-fonte deste Tutorial foi compilado em \DTMportugesmonthname{\the\month} de \the\year.\par

\backmatter

\printindex % imprime a página do índice.

\newpage
\nopagecolor

\begin{center}
	\AddToShipoutPictureBG*{\includegraphics[width=\paperwidth,height=\paperheight]{airplane.jpg}}
	\thispagestyle{empty}
	\renewcommand{\thepage}{fim}
\end{center}

\end{document}
